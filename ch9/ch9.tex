\chapter{Conclusion}
\section{Summary of Results}
To conclude, it is known that a neutrino has mass and propagates in mass eigenstates; however, the individual masses are unknown since current experiments can only probe the mass difference. The theory for how neutrinos get their mass suggests that neutrinos could be a Majorana particle. To experimentally verify this fact, one can look at double beta decay and theorize that there could be a virtual neutrino exchange such that no neutrino is emitted, and the other emitted particles would carry their energies. HPGe provides a compelling approach to look for neutrinoless double beta decay. The signal formation is very well understood and can be modeled using simulations.

LEGEND collaboration is looking for {\onbb} in Ge-76 using a phased approach and with a goal to probe the bottom of the inverted hierarchy. Surface backgrounds are the largest components in the background model and a significant source of uncertainty due to the lack of proper modeling. Surface backgrounds are an important component in the LEGEND background model and are difficult to model. We developed a new method that can model these challenging backgrounds.

EH-Drift extends the capabilities of existing HPGe detector simulations by incorporating detailed physical processes such as diffusion, self-repulsion, and surface charge effects, all while maintaining computational efficiency through GPU acceleration. Its ability to model large charge cloud drifts and surface drifts makes it a valuable tool for understanding detector response to surface events and improving background modeling in experiments.


To conclude, we developed the Cyclic Positional U-Net that can perform ad-hoc pulse shape translations without modeling any of those effects. This model trains on arbitrarily collected and unpaired datasets of simulated and detector pulses. CPU-Net successfully translates simulated pulses to outputs indistinguishable from actual detector pulses by applying corrections according to proper detector physics. Furthermore, we showed that the CPU-Net correctly reproduces the distribution of two critical pulse shape reconstruction parameters. 


\section{Future Direction}
\begin{itemize}
    \item improved neural network wasserstein metric
    \item improve interpolation for ehdrift maps
    \item Surface drift in SSD and perfoming drift and diffusion on GPU in SSD
\end{itemize}
\label{chap:conclusion}