\chapter{Conclusion}
\label{chap:conclusion}
\section{Summary of Results}
Neutrinos are fundamental to understanding the matter-antimatter asymmetry in the universe. The theory of how neutrinos get their mass suggests that neutrinos could be Majorana particles. To experimentally verify this fact, one can search for neutrinoless double beta decay, which can occur if neutrinos are Majorana particles. HPGe detectors provide a compelling approach for looking for neutrinoless double-beta decay, since they have excellent energy resolution and high internal purity. Signal formation in these detectors is very well understood and can be modeled using simulations. The LEGEND Collaboration is seeking {\onbb} in Ge-76 using a phased approach and with the goal of investigating the full inverted hierarchy parameter space of $m_{\beta \beta}$. Pulse shape simulations that model signal formation in the detector help us to understand the background model components and simulate the statistical methods used to reject background such as multi-site and surface events. Reliable simulations have been hindered by a lack of modeling of surface events and replication of the effects of the experiment's electronics response.

Surface backgrounds are potentially the largest components for the future LEGEND-1000 experiment and a significant source of uncertainty due to the lack of proper modeling of charge collection. We developed a new method that can model these challenging backgrounds on the passivated surfaces of detectors. {\ehd} extends the capabilities of existing HPGe detector simulations by incorporating detailed processes such as diffusion, self-repulsion, surface charge, and reduced surface mobility. Its ability to model large charge cloud densities and surface drift makes it a valuable tool for understanding detector response to surface events and improving background modeling in experiments. Using GPU computing, the simulations can be significantly accelerated, allowing for a large-scale background modeling study. Using activeness maps created using {\ehd} we were able to reproduce the measurements made by previous test stands that measured the detector response to alphas at varying positions on the detector surface. We also showed that maps can be used to calculate the active volume for {\onbb} events efficiency and to create spectral models for alpha and beta events incident on the passivated surface of HPGe detectors.

Another challenge in accurately simulating HPGe detector waveforms is to correctly model the effects of the readout electronics. It is quite challenging to use first-principles corrections to fully account for the electronics response. We developed a CycleGAN network that learns to translate simulated pulse shapes into data-like signals without explicitly modeling any of the electronics. The model combines a positional U-Net generator with an RNN-based discriminator. This model trains on unpaired data sets of simulated and detector pulses that are collected during routine calibration measurements. {\cpunet} successfully translates simulated pulses to output waveforms that closely resemble actual detector pulses by applying corrections using a neural network.  We show that {\cpunet} preserves the necessary physics in the waveforms and correctly reproduces the critical pulse shape parameters.

\section{Future Directions}
This work has laid the foundation for improving pulse shape simulations in LEGEND by addressing some longstanding challenges in modeling surface events and the detector electronics response. Looking ahead, there are several avenues for improvement in the two models. The first step would be a detailed validation of the results, which would allow for refining the models.

\subsection{Validation of Results}
To validate the results of {\ehd}, a direct comparison of the surface event waveforms with {\ehd} output can be performed. This will help validate the results while providing inputs to better determine tunable parameters such as surface drift velocity, passivated surface depth, etc. A surface scanning cryostat is currently being commissioned at UNC and will provide additional data needed to validate the simulations. The scanner aims to perform multidetector alpha measurements with a $^{241}$ Am source and a monoenergetic beta study with a $^{137}$Cs source. Waveforms from the scanner can be used to validate the simulations as the energy and location of events are known from the source type and location. The ability of the scanner to perform rapid scanning of different detectors will help us determine the detector-to-detector variation in the surface effects. This will also help us understand how these values depend on event location, energy deposition, alpha incidence angle and detector temperature.

The Compton scanner in TUM provides an opportunity to validate the {\cpunet} results \cite{Abt_2022odr}. The scanner operates by irradiating the detector with gamma rays and then collecting the scattered gamma rays with pixelated detectors. Using the Compton camera technique, these measurements can be used to create a library of data waveforms with known positions. We can use this paired dataset to train and validate {\cpunet} since we can now perform a direct comparison of ATN output with data waveforms.

\subsection{Improving {\ehd} Activeness Maps}
Activeness maps in {\ehd} have proven useful for modeling surface effects while reducing simulation runtime by interpolating between points. Until now, we have been using cubic interpolation to build activeness maps in {\ehd}. Although this method is a good approximation, it does not always capture the sudden changes in activeness near the passivated surface. This is because the charge-collecting efficiency varies much more rapidly in the near-surface areas where surface effects dominate than in the bulk of the detector. Thus, the cubic interpolation results tend to have too sharp transitions close to the surface. One way to address this is to adopt interpolation that accounts for local gradients in the activeness function. Techniques like radial basis function interpolation could be used to refine the interpolation in high-variation regions while maintaining a coarser interpolation in the bulk. Another possibility is to perform future studies on the activeness to determine the optimal sampling of the detector. The electric field near the surface with detector type, energy, and surface charge can give a relation for how the activeness is going to change, and this can be used to sample each detector optimally.

\subsection{Future for Pulse Shape Simulations}
The future development in pulse shape simulations for Germanium detectors will be centered around {\ssd} . Although {\ehd} successfully captures diffusion, self-repulsion, and surface charge effects, it also makes several assumptions in 2-D simulations of a 3-D problem that {\ssd} can, in principle, avoid. The fully three-dimensional approach of {\ssd} provides a more complete picture of electric fields and charge drift, and it also provides the ability to model the effects of the surroundings of a detector. The program is written in \texttt{Julia} which provides a natural way to parallelize the calculations.

The techniques developed in {\ehd} such as GPU-based drift and diffusion and surface drift can be used to improve {\ssd}. Although {\ssd} already uses GPUs for field calculations, performing drift and diffusion on GPUs will enable simulations using large charge cloud surface events more efficiently in {\ssd}. In {\ssd} the number of charges used is an important input. The higher the number of charges, the better the results are, but after a certain number, there is a diminishing return in improvement. A study comparing {\ehd} results and {\ssd} can help find this optimal point to get accurate simulations while optimizing runtime. We performed some initial {\ssd} simulations with surface drift effects included, and the results looked promising.

\subsection{Challenges in CycleGAN based Model}
The {\cpunet} was quite effective at translating simulated waveforms to data-like pulses, while matching the current amplitude and drift time distribution. However, the CycleGAN training can be quite unstable and requires fine-tuning of several hyperparameters to get the right balance. This can be seen in figure \ref{ch8_fig_tail_slope_sim}, where the distribution of data and the ATN output waveform do not match. We also noticed that the noise added by {\cpunet} is not always consistent with what we see in the data. 

To test both effects, we added pole-zero decay and pink noise to the simulated waveform. This gave us a direct one-to-one data set for validation. We trained {\cpunet} with simulations plus electronics effects as data. After training, we observed that while the ATN output waveform resembled the data waveform better, there was still a significant difference between them. This shows that the model learns the electronics response to an extent to fool the discriminator, but not necessarily to the high precision we would like to achieve.

\subsection{{\cpunet} Model Improvements}
One way to improve the model is to incorporate distribution-based loss metrics, such as the Wasserstein distance, into loss optimization \cite{vaserstein1969markov}. This can be done for known parameters such as the tail slope, current amplitude, and drift time. This forces the model to learn the correct distribution during training. We would not know how the model works for unknown electronics parameters; however, proving the known parameters can give the model a first guess to improve upon. Combining the Wasserstein metrics with cycle consistency and identity consistency could help the generator learn the features more effectively. Input simulations can also be improved by using {\ssd} instead of siggen, and using SSD's capability to model charge cloud effects more accurately. This will provide better signal modeling and initial approximation for drift time and current amplitude distributions.

Another approach would be to replace CycleGAN with an advanced generative model such as the diffusion model \cite{2020arXiv200611239H} or the Transformer model \cite{Transformer}. The diffusion model gradually adds Gaussian noise to the data pulse in a Markov chain. Then it trains a network to reverse the chain. This allows for finer control over the learned distribution. The advantage of the method is that we can control the noise being encoded. Thus, using known waveform shape parameter priors, such as tail slope, current amplitude, and drift times, we could guide the model to learn the correct translations of simulated pulses.

\section{Implications for the LEGEND Physics Program}
Together, the two methods improve the accuracy of pulse shape simulations. This will enable a large-scale simulation study of LEGEND backgrounds, tests of new pulse shape discrimination strategies, and allow simulations to be adapted to reflect different detector configurations.  

\subsection{Passivated Surface Background Modeling for LEGEND}
Having an accurate model will enable us to create a background model for events on the passivated surfaces that correctly accounts for the anomalous charge collection effects in this region. This will impact the predicted spectral shape of all background components, particularly in detectors like PPCs that have large passivated surfaces, with particularly large impacts on the backgrounds associated with surface events such as alpha and beta interactions. Using \texttt{Geant4} simulations, we can estimate the location and energy of interactions near the passivated surface from varying background sources. Then, using libraries of {\ehd} activeness maps with different surface charges, we can apply the charge collection efficiency determined for each location to estimate the energy spectrum for these depositions. Ultimately, this would enable us to create a background model that accurately describes the behavior of the passivated surface events. Knowing the fraction of those events appearing in the {\onbb} ROI would help reduce the uncertainty in the alpha background contribution in the LEGEND experiment, improve models for the shape of the K-42 spectrum, and improve estimates of the signal efficiency in detectors with large passivated surfaces.

\subsection{Beyond Standard Model Searches}
The {\ehd} may also have an important application in low energy studies. Enriched detectors in {\Lthou} will have limited cosmogenic exposure on the surface and a very low $^{39}$Ar background event rate, resulting in ultra-low backgrounds at low energy. This will enable highly sensitive searches for other physics Beyond the Standard Model (BSM) such as signatures of light weakly interacting massive particles (WIMP) and bosonic dark matter (DM). Low-energy BSM physics searches would require an accurate model of the passivated surface. This is because most background events at low energy happen on the surface. A model for these passivated surface interactions would directly improve the low-energy sensitivity in {\Lthou} by improving background modeling near threshold. BSM effects can alter the shape of $2\nu\beta\beta$, and studying the distribution can set limits or make a measurement of BSM processes like Majoron emission. {\ehd} will play a crucial role in modeling $2\nu\beta\beta$ spectral distortions since such studies would require detailed modeling of the background of $^{39}$ Ar beta, a passivated surface effect and of how bulk events in the vicinity of the passivated surface are degraded in energy.

\subsection{LegendGeSim and {\cpunet} Integration}
The LEGEND Germanium Simulation Chain (LegendGeSim) is a simulation pipeline that can be used to simulate the entire electronics chain from geometry visualization to the simulation of raw files with realistic data that mimic the data. The goal is to produce a set of simulated waveforms that are compatible with tools used to process data waveforms. {\cpunet} will be incorporated into the LegendGeSim raw tier to help add electronics effects to the data. This will enable more realistic pulse shape simulations and enable calculating the efficiency of cuts by directly applying them to the data.