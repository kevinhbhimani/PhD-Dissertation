\begin{table}[!htb]
    \centering
    \renewcommand{\arraystretch}{1.3} % Adjust row spacing
    \begin{tabular}{|l|p{10cm}|c|}
        \hline
        \textbf{Flag} & \textbf{Description} & \textbf{Example} \\ 
        \hline
        -r & Set the radial position of the event in mm. & 15.00 \\
        \hline
        -z & Set the axial position of the event in mm. & 0.50 \\
        \hline
        -g & Specify the detector name. & P42575A \\
        \hline
        -s & Set the surface charge density in $10^{10} e/\text{cm}^2$. & -0.50 \\
        \hline
        -e & Input the interaction energy in keV. & 5000 \\
        \hline
        -v & Choose whether to write density files (0 = no, 1 = yes). & 1 \\
        \hline
        -f & Choose whether to recalculate the electric field at each time step (0 = no, 1 = yes). & 1 \\
        \hline
        -w & Choose whether to save the electric field at the end of the program (0 = no, 1 = yes). & 1 \\
        \hline
        -d & Choose whether to save the depletion surface (0 = no, 1 = yes). & 1 \\
        \hline
        -p & Choose whether to write the weighting potential (0 = no, 1 = yes). & 1 \\
        \hline
        -b & Set the bias voltage in volts. & 3500 \\
        \hline
        -h & Specify the grid size in mm. & 0.0200 \\
        \hline
        -m & Define the passivated surface depth in mm. & 0.00020 \\
        \hline
        -c & Set the velocity of surface charges compared to bulk. & 0.001 \\
        \hline
        -a & Input a custom impurity density profile file. & filename.dat \\
        \hline
        -t & Define the waveform length to be simulated, in ns. & 16000 \\
        \hline
        -u & Set the frequency of output signal saving in ns. & 16 \\
        \hline
    \end{tabular}
    \caption{Input parameters to EH-Drift}
    \label{ch3_tab_ehdrift_parameters}
\end{table}
