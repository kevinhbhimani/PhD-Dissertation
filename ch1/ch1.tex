\chapter{Neutrinoless Double Beta Decay}
In the early parts of the twentieth century, physicists were surprised to find that the electron in beta decay seemed to violate momentum conservation. Instead of carrying the full Q-value of the decay, the electron carried a range of energies. To explain this anomaly, Wolfgang Pauli in 1930 theorized a particle as a desperate remedy to save conservation of momentum. It was called the neutrino because it was electrically neutral and because its rest mass was thought to be zero. It took 20 years to detect these mystery particles, and since then, neutrinos have continued to surprise physicists. Initially, neutrinos were thought to be massless, but experiments have demonstrated that neutrinos have a very small mass.

% It has a spin of $\frac{1}{2}$ (thus a fermion). The neutrino was originally thought to be massless. 
\begin{figure}
\centering
\includegraphics[scale=0.4]{ch1/figs/0nbb_feynman_diag.png}
\caption{Comparison of the masses of fermions in the Standard Model. Neutrino masses were estimated assuming the normal mass hierarchy and a loose upper bound \cite{Hewett:2012ns}.}
\label{n_mass_comp}
\end{figure}

Through the development of the theory and experimental observation of neutrinos, three favors of neutrinos have been discovered: electron ($\nu_e$), muon ($\nu_{\mu}$), and ($\nu_{\tau}$): corresponding to the three generations of charged leptons they are associated with. Experimental limits on the masses of these particles are orders of magnitude smaller than those of all other standard model fermions as shown in Fig. \ref{n_mass_comp}. Each neutrino flavor state is a linear combination of three discrete mass eigenstates, causing neutrinos to oscillate between those flavors as they travel through space \cite{Super_Kamiokande_1998kpq}. To this date, it is not known what the absolute masses of neutrinos are, and why their masses are so small. Neutrinos are hard to study as they are only theorized to interact via gravity and weak interaction. Most neutrino experiments are based on studying weak neutrino interactions, as the small mass makes gravitational neutrino studies practically impossible. We thus begin with a brief review of weak interactions.


\section{Weak interactions}
The weak interaction is one of the four fundamental forces described in the standard model. It has a short effective range and is responsible for the radioactive decay of nuclei. In weak interaction, fermions can exchange three types of force carriers: W$^+$, W$^-$ and $Z$ bosons. The interactions associated with the charged W$^+$ and W$^-$ bosons are known as charged current interactions, while those associated with neutral $Z$ boson are known as neutral current interactions. Fig. \ref{fig:weak_current} shows the vertices for each process. Together these vertices form the basis for weak interactions between leptons in the standard model. Weak interaction does not preserve parity-symmetry.\cite{wu_experiment}

\begin{figure}
\centering
\includegraphics[width=0.8\linewidth]{ch1/figs/weak_current.png}
\caption{Feynman diagram showing vertices for three different kind of weak interaction mechanisms. \cite{ParticleDataGroup:2014cgo}}
\label{fig:weak_current}
\end{figure}


% Depending on the charge and mechanism of the interacting boson, there could be different particles formed from the decay. This can be understood by considering two kinds of beta decay. A beta-minus decay with no initial charge converts a neutron into a proton in the same vertex. Thus the interacting boson has to carry a negative charge and produces an electron in that vertex. Similarly in a beta-plus decay, the boson is $W^+$ and thus a position is formed. The Feynman diagram for two kinds of Beta decay is demonstrated in Fig. \ref{beta_decay_types}.

% \begin{figure}
% \centering
% \includegraphics[scale=0.35]{Theory/figs/bmibplus_decay.png}
% \caption{Two kinds of beta decay demonstrating difference mechanism of interaction leading to different types of beta decay. \cite{Meijer:2019qvc}}
% \label{beta_decay_types}
% \end{figure}

% There are three neutrino flavors based on the three charged leptons (electrons, muons, and tau) and their corresponding antineutrinos. Neutrinos are created based on an intermediate boson that carries the interaction. For instance, a process releasing an electron would also require the emission of an electron antineutrino. 

In particle physics, the Lepton number accounts for the net number of leptons in a reaction. Particles are given a unit positive lepton number and anti-particles a unit negative. Two emergent properties from the Standard Model of Physics are that the lepton flavor and total lepton number L, are conserved quantities. The detection of neutrino oscillations revealed that the lepton flavor is not conserved and suggested that there might be physics beyond the standard model (BSM). The mechanism by which the neutrino gets its mass might provide an instance of lepton number violation, and provide additional evidence for BSM physics.

%In the standard model of physics, the Lepton number is said to be conserved. Lepton number conservation does not have a theoretical motivation and is not associated with any gauge symmetry. 
\section{Neutrino Mass}

\subsection{Dirac Mass Mechanism}
There is no mechanism in the standard model to give neutrino mass. One way to add a mass would be to use four neutrino fields that couple via the Higgs mechanism. The Higgs mechanism requires the existence of a right-handed neutrino field because, in the standard model, interaction with the Higgs boson couples opposite-chirality fields. The weak interaction is parity-violating and thus only the left-handed neutrino and right-handed anti-neutrino are produced or observed via interactions. They alone cannot couple to the Higgs field. It is possible to consider the right-handed neutrino and left-handed anti-neutrino to be sterile to weak interaction, but this would not explain the relatively low observed neutrino mass limits.

\subsection{Majorana Mass Mechanism}
A different approach would be for the neutrino to be a Majorana fermion. Then the right-handed antineutrino, $\bar{\nu_R}$ could be coupled to the left-handed neutrino, $\nu_L$. Thus, the left-chiral field would correspond to the neutrino and the right-chiral field would correspond to the antineutrino, and the neutrino and antineutrino could only be distinguished by chirality. This mechanism was first postulated by the Italian physicist Ettore Majorana \cite{Majorana_1937}. Along with the See-Saw mechanism, it could provide a basis for neutrino mass, and also explain why neutrinos are much lighter than any other standard model particles. The Majorana nature of the neutrino could lead to a lepton number violation.


\subsection{See-Saw Mechanism}
The See-Saw mechanism is a model that attempts to explain the relatively low observed neutrino mass limits by including both Dirac and Majorana masses in the Lagrangian. This suggests that there would be two light neutrinos and two heavy neutrinos terms such that the Lagrangian is given by:


% As first described in \cite{see-saw_article}, the Dirac mass terms have the form of:
% \begin{equation}\label{dirac_mass_terms}
% -m_D(\bar{\nu}_L\nu_R+\bar{\nu}_R\nu_L)
% \end{equation}

% and the Majorana term will have a form of:
% \begin{equation}\label{majorana_mass_terms}
% -\frac{1}{2}m^L_M(\bar{\nu}_L\nu_L^c+\bar{\nu}_L^c\nu_L)-\frac{1}{2}m^R_M(\bar{\nu}_R\nu_R^c+\bar{\nu}_R^c\nu_R)
% \end{equation}

% Using charge conjugation operator C, charge conjugation of the field can be expressed as:
% \begin{equation}\label{charge_conjugation1}
% \nu^C=C\nu=i\gamma^2\nu^*
% \end{equation}
% and
% \begin{equation}\label{charge_conjugation2}
% \bar{\nu}^C=\nu^TC=\nu^Ti\gamma^2
% \end{equation}

\begin{equation}\label{mass_lagrangian}
\mathcal{L}_{mass} = -\frac{1}{2}(\bar{\nu}_L\bar{\nu}_R^c) \begin{bmatrix} m_M^L & m_D \\ m_D & m_M^R \end{bmatrix} \begin{bmatrix} \nu_L^c \\ \nu_R \end{bmatrix} + hc
\end{equation}

$m_D$ is the Dirac mass term, $m_M^L$ and $m_M^R$ are left and right-handed Majorana mass terms, and $hc$ is the Hermitian conjugate of the prior term. Left-handed Majorana mass term is forbidden by $SU (2)$ symmetry so we let $m_M^L \rightarrow 0$. Diagonalizing the resulting matrix gives two effective fields: 

\begin{equation}\label{mass_lagrangian_reduced}
\mathcal{L}_{mass} = -\frac{1}{2}(\bar{\nu}_L\bar{N}) \begin{bmatrix} m_\nu & 0 \\ 0 & M_N \end{bmatrix} \begin{bmatrix} \nu \\ N \end{bmatrix} + hc
\end{equation}

The resulting eigenvalues are $M_\nu \approx \frac{m^2_D}{m_R}$ and $M_N \approx m_R$. We have thus identified the mass eigenstates for neutrinos. Note that if one of the mass terms goes up, the other has to go down, explaining the origin of the `See-Saw' nomenclature. This would explain the very small mass of observed neutrinos. The left-handed light neutrinos and right-handed light antineutrinos would be the ones observed in current experiments. The heavy right-handed neutrinos and left-handed antineutrinos would have a huge phase space, thus a shorter half-life, and would decay quickly in the early universe.

Majorana neutrinos could also explain leptogenesis, a leading model to account for the excess of matter over antimatter in the observed universe. During the Big Bang, the same amount of matter and anti-matter should have been created. However, the present universe is dominated by matter. Majorana neutrinos would account for this excess of matter over antimatter in the observable universe. Experimentally detecting Majorana neutrinos is of great interest since it would provide a potential path to matter-antimatter symmetry in the universe along with an instance of lepton number violation-- a direct observation of physics beyond the standard model.
%More instances of lepton number violations would further destabilize the standard models, so experiments looking for such violations are highly motivated.

\subsection{Measuring neutrino mass}

There are four ways to study the neutrino mass. The first is using cosmology. The Lambda-CDM model-based spectral fit to the Plank data and other observations of large-scale structures in the universe can be used to measure the sum of the neutrino masses. The sum of masses is one of the model's free parameters, and the current limit implies a $\Sigma m_\nu<0.23$ eV \cite{Planck_2015fie}. This sum is, however, model-dependent and relies on an extensive understanding of how the universe evolved. 

A model-independent approach is to measure the mass of a neutrino kinematically. This is done by studying the endpoint of a beta decay spectrum, which would have a slight variation in shape depending on the neutrino mass. The KATRIN collaboration is performing such direct mass measurement by looking at the tritium $\beta$-decay spectrum and has achieved a sensitivity in $m_\beta < 0.7 \text{ eVc}^{-2}$ at a $90\%$ confidence level \cite{KATRIN:2022}.

The third approach is to use neutrino oscillations experiments though these can only set a lower limit on $\Sigma m_\nu$ or $m_\beta$. The mass eigenstates of neutrino form a complete, orthonormal basis. Similarly the three flavors $\nu_e$, $\nu_\mu$ and $\nu_\tau$ form yet another orthonormal base. Experiments indicate that these two eigenbases are rotated relative to each other. The Pontecorvo–Maki–Nakagawa–Sakata matrix represents the unitary transformation between the two bases:

\begin{equation}\label{pmns_matrix}
\begin{bmatrix} \nu_e \\ \nu_\mu \\ \nu_\tau \end{bmatrix} = \begin{bmatrix} U_{e1} & U_{e2} & U_{e3}\\ U_{\mu 1} & U_{\mu 2} & U_{\mu 3} \\ U_{\tau 1} & U_{\tau 2} & U_{\tau 3} \end{bmatrix} \begin{bmatrix} \nu_1 \\ \nu_2 \\ \nu_3 \end{bmatrix}
\end{equation}

Eq. \ref{pmns_matrix} is often expressed using three mixing angels ($\theta_{1,2}$, $\theta_{2,3}$ and $\theta_{1,3}$) and three phase angles ($\delta_{CP}$, $\alpha_1$ and $\alpha_2$).

\begin{equation}\label{pmns_matrix_expanded}
U = \begin{bmatrix} 1 & 0 & 0\\ 0 & c_{23} & s_{23} \\ 0 & -s_{23} & c_{23} \end{bmatrix} \begin{bmatrix} c_{13} & 0 & s_{23}e^{-i\delta_{cp}}\\ 0 & 1 & 0 \\ -s_{23}e^{-i\delta_{cp}} & 0 & c_{13} \end{bmatrix} \begin{bmatrix} c_{12} & s_{12} & 0\\ -s_{12} & -c_{12} & 0 \\ 0 & 0 & 1 \end{bmatrix} \begin{bmatrix} 1 & 0 & 0\\ 0 & e^{i\frac{\alpha_1}{2}} & 0 \\ 0 & 0 & e^{i\frac{\alpha_2}{2}}\end{bmatrix}
\end{equation}

where $c_{ij}=\mathrm{cos}(\theta_{i,j})$ and $s_{ij}=\mathrm{sin}(\theta_{i,j})$. This gives the probability of a neutrino of flavor $\alpha$ to oscillate to flavor $\beta$ as:

\begin{equation}\label{n_oscillation}
P_{\alpha\rightarrow\beta} = \left|\sum_i U^*_{\alpha i}U_{\beta i} e^{-i\frac{m_i^2L}{2E}}\right|^2
\end{equation}

It can be shown that Eq. \ref{n_oscillation} depends on $\Delta m^2_{i,j}=m^2_i-m^2_j$ and not on individual masses. Thus, neutrino oscillation experiments can determine the squared difference of the masses and their sign, but not the individual masses of different flavors of neutrinos. We know from experiments such as SNO that $m_2>m_1$; however, no information is available about the sign of $\Delta m_{23}^2$. Therefore, it is not known which neutrino is the heaviest and what the value of each mass eigenstate is. This problem is called the neutrino mass hierarchy problem. If mass $m_3$ is heavier than mass $m_2$, the hierarchy is called normal, but if it is lighter, the hierarchy is said to be inverted, as shown in Fig. \ref{mass_hierarchies_fig}.

\begin{figure}%[H]
\centering
\includegraphics[scale=0.25]{ch1/figs/mass_hierarchies.png}
\caption{Mass hierarchies for neutrinos \cite{Hewett:2012ns}.}
\label{mass_hierarchies_fig}
\end{figure}

The fourth approach is to study the double-beta decay which would probe the effective $\langle m_{\beta\beta}\rangle$ neutrino mass term.

\section{Double Beta Decay}
Some nuclei need to undergo a beta-decay to move close to the ideal ratio of protons to neutrons; however, they are energetically forbidden to undergo such a decay. These nuclei may instead undergo double beta decay to achieve the optimal ratio of nucleons. In double beta decay, two neutrons in the nucleus are converted to protons simultaneously, producing two electrons and two electron antineutrinos. It is represented by the Eq. \ref{beta_decay_eq}. It was first discussed by M. Goeppert-Mayer and there are 35 potential nuclei in nature that can undergo this process\cite{ZUBER_2012}.

\begin{equation}\label{beta_decay_eq}
(Z,A) \rightarrow (Z+2,A) + 2e^- + 2\bar{\nu}_e
\end{equation}

The necessary conditions for a $\beta$ decay are shown in Fig. \ref{2nbb_cond}. Only isotopes on the lower parabola with even-even nuclei are allowed to undergo double beta decay. The single beta decay must also be forbidden, and the decay must go through two subsequent steps. Since such simultaneous decay of two nucleons in the same nucleus is very unlikely, double beta decay is a rare process, but has remarkably been observed experimentally in $^{82}$Se with a half-life of $(1.1^{+0.8}_{-0.3})\times 10^{20}$ years by S. Elliott et al. \cite{PhysRevLett.59.2020}. It has been observed in more nuclei since then.

\begin{figure}
\centering
\includegraphics[width=0.6\linewidth]{ch1/figs/2nbb_cond.png}
\caption{Ground state mass parabola for nuclei showing required conditions to undergo a double beta decay. \cite{2nbb_cond}}
\label{2nbb_cond}
\end{figure}

\section{Neutrinoless Double-Beta Decay}
If the neutrino were a Majorana particle, it would be possible to have a double beta decay without the emission of the two neutrinos. Instead, the neutrino is exchanged as a virtual particle such that one nucleon absorbs the neutrino emitted by another. This process is called neutrinoless double-beta decay. This scenario, which is considered the simplest possible model, is termed ``light neutrino mediated'' decay.


% \begin{figure}[htb]
% \centering
% \includegraphics[scale=0.25]{Theory/figs/0nbb_feynman_diag.png}
% \caption{Feynman Diagram for light neutrino exchange Neutrinoless Double Beta Decay process. \cite{Bilenky:2014uka}}
% \label{0nbb_feynman_diag}
% \end{figure}

The overall reaction would be:

\begin{equation}\label{0nbeta_decay_eq}
(Z,A) \rightarrow (Z+2,A) + 2e^-
\end{equation}

An observable quantity of the process is the decay rate. Using Fermi's golden rule it can be expressed as \cite{mjd2013}:
\begin{equation}\label{0nbbdecay_rate}
(T^{0\nu}_{1/2})^{-1} = G^{0\nu}\left|M_{0\nu}\right|^2\langle m_{\beta\beta}\rangle^2
\end{equation}

The phase space term $G^{0\nu}$ corresponds to a phase space into which the electron can decay. Phase space can be calculated exactly for any given isotope. $M_{0\nu}$ is the nuclear matrix element (also called ``transition amplitude''). It encapsulates all of the nuclear physics processes occurring inside the nucleus and can be understood as the probability of the transition between the initial and daughter nuclei. It is difficult to know the initial and final wave function of the nuclei, so calculating nuclear matrix elements is non-trivial and is an active field of research \cite{Menendez:2017fdf}. Finally, $\langle m_{\beta\beta}\rangle$ is the effective neutrino mass term and can be represented as:

\begin{equation}\label{effective_mjd_mass}
\langle m_{\beta\beta}\rangle =  \left|\sum_{i=1}^{3} U^2_{ei}m_i\right|
\end{equation}

$U_{ei}$ are the components of PMNS neutrino mixing. This process described above assumes an exchange of a light Majorana neutrino, but there can be other exotic physics that can contribute to the decay rate (see \cite{Schechter_1982} for an example). However, all of these processes would still imply that the neutrino is a Majorana particle.

Eq. \ref{0nbbdecay_rate} depends on the mass hierarchy and can help probe the absolute neutrino mass scale. The observed decay rate would be different depending on the neutrino mass hierarchy. The effect can be seen in Fig. \ref{majorana_mass}. If neutrino masses are large compared to the size of the mass splittings, we do not expect a significant effect on $m_{\beta\beta}$ as both hierarchies merge at higher masses; however, if the neutrino masses are similar in magnitude to the mass splittings, there would be a difference in the $m_{\beta\beta}$ based on the hierarchy. {\onbb} studies provide a fourth and unique way to understand neutrino mass as they probe the effective majorana mass $\langle m_{\beta\beta}\rangle$, and provide a direct way to measure the ordering of neutrino mass. Detecting this extremely rare process, however, requires meticulous background reduction and large-mass experiments.
\begin{figure}
\centering
\includegraphics[width=\linewidth]{ch1/figs/effective_maronana_mass.png}
\caption{Marginalized posterior distributions for $m_{\beta\beta}$ and $m_l$ for the normal (right) and inverted (left) mass ordering \cite{PhysRevD.96.053001}. Plotted assuming QRPA NMEs and the absence of mechanisms that drive $m_{\beta\beta}$ and $m_l$ to zero. The allowed parameter space assuming $3\sigma$ intervals of the neutrino oscillation observables from NuFIT is shown by the solid solid lines \cite{nufit}.}
\label{majorana_mass}
\end{figure}


\section{Detecting Neutrinoless Double Beta Decay}
Numerous experiments have searched for neutrinoless double-beta decay ($0\nu\beta\beta$), and many are presently searching for it. The measurement of interest is the sum of the kinetic energies of the two emitted electrons. If no antineutrino is emitted, the emitted electrons should carry the full energy difference between the final and initial nuclear state (the Q value of the reaction). The experimental signature of such a decay would be a mono-energetic peak in energy at the endpoint of the two-neutrino spectrum, indicating no antineutrinos are emitted, as shown in Fig. \ref{mjd_background_fig} for ${}^{76}\mathrm{Ge}$ in {\MJD}. Most $0\nu\beta\beta$ experiments involve having a source of double beta decay that acts as both a source and a detector. The region of interest (ROI) is defined as the narrow energy window around the $Q_{\beta\beta}$ and is selected according to the energy resolution of the detector. Using radioactive decay theory, the number $0\nu\beta\beta$ candidate events observed in the ROI is given by: %\cite{annurev_nucl}

\begin{equation}\label{roi_candidate}
N=\ln(2)\frac{N_A}{W}\left(\frac{a\epsilon MT}{T^{0\nu}_{1/2}}\right)
\end{equation}
where $N_A$ is the Avogadro's number, $W$ is the molar mass of the source, $\epsilon$ is the detector efficiency of the signal in ROI, and $t$ is the observed time. Using Eq. \ref{roi_candidate}, we can understand how the half-life sensitivity would depend on the detector:

\begin{equation}\label{det_sensitivity}
T^{0\nu}_{1/2} \propto
    \begin{cases}
    aM\epsilon T & \text{(background free)}\\
    a\epsilon\sqrt{\frac{MT}{B\Delta E}} & \text{(with background)}
    \end{cases}       
\end{equation}
such that $\Delta E$ is the energy resolution of the detector, and B is the background index of the experiment normalized to the width of ROI, source mass, and measurement time. The importance of low background is apparent as $T^{0\nu}_{1/2}$ scales linearly with run time t for no background experiment compared to $\sqrt{t}$ in presence of backgrounds.


\begin{figure}
\centering
\includegraphics[scale=0.5]{ch1/figs/DoubleBetaEnergy.png}
\caption{Simulated $0\nu\beta\beta$ signal in a MAJORANA detector. \cite{mjd_background}}
\label{mjd_background_fig}
\end{figure}

Current experiments have set limits on $T^{0\nu}_{1/2}$ half-life greater than $10^{26}$ years \cite{KamLANDZen2018}.  For a substantial probability of discovery, the experiments looking for $0\nu\beta\beta$ need to have a source as large as possible with extremely low backgrounds. At that sensitivity level, natural isotopes such $^{222}$Rn, $^{232}$Th, and $^{232}$U can introduce background events. Comic rays are another potential source, along with the $2\nu\beta\beta$ decay depending on the energy resolution of the experiment. Most of the experiments are thus housed underground to decrease the rate of cosmic-ray-induced backgrounds and cosmogenic activation of detector materials. They are also heavily shielded, and the experiment parts are ultra-clean. The experimental programs also need to have extensive radioactive assay studies to understand the background of their experiments.

\begin{figure}
\centering
\includegraphics[width=0.6\linewidth]{ch1/figs/exposure_plot.png}
\caption{The background requirements and exposure needed for 3$\sigma$ confidence level discovery of neutrinoless double beta decay (in over $50\%$ of a collection of identical experiments), given the inverted hierarchy \cite{Gruszko:2017kfx}.}
\label{exposure_plot}
\end{figure}

The $\beta\beta$ isotope used must ideally have a high Q value and a low double beta decay rate. It also must be able to be made into big detectors to increase the probability of discovery. Several isotopes are used by experiments searching for neutrinoless double-beta decay, including ${}^{76}\mathrm{Ge}$ (GERDA\cite{GERDA_final}, {\MJD}\cite{mjd2013}) ${}^{136}\mathrm{Xe}$, (KamLAND-Zen\cite{KamLANDZen2018}, EXO\cite{Auger2012ar}), and ${}^{130}\mathrm{Te}$ (CUORE\cite{Arnaboldi2002du}, SNO +\cite{SNO_paper}). 


In the coming years, the next generation of experiments will attempt to explore the entire inverted hierarchy region of Fig. \ref{majorana_mass}. Fig. \ref{exposure_plot} shows the exposure needed to achieve a $3\sigma$ discovery at the bottom of the inverted hierarchy region in ${}^{76}Ge$. The different blue lines represent background levels in counts per Region of Interest (ROI). An ideal experiment would strive to eliminate all of its background in ROI and follow the solid blue line. As the experiments increase their exposure time, the sensitivity will increase and reach the bottom of the inverted hierarchy band. The bottom of the band is shown with the shaded blue box, which is calculated using various nuclear matrix element calculations. Next-generation experiments would test inverted hierarchy space and provide new insight into the nature of neutrinos. In the next section, we explore ${}^{76}\mathrm{Ge}$ as a choice of the detector and the LEGEND experiment's search for neutrinoless double-beta decay.

% Thus the next decade would be an exciting time for research in neutrino physics. With ton scale experiments such as LEGEND planned 1000 \cite{legend2017}, there might be major revelations about the properties of the neutrino. We might find that neutrinos are their antiparticles, or at least that inverted ordering is not true. In that case, new experiments and possible theories must be postulated to explain neutrino mass and its interaction. With the recent discovery of an anomaly in the magnetic moment of muons, the discovery of neutrinoless double beta decay will further break the standard model of physics.

% To conclude, it is known that a neutrino has mass and propagates in mass eigenstates; however, the individual masses are unknown since current experiments can only probe the mass difference. The theory for how neutrino gets their mass suggests that neutrinos could be a Majorana particle. To experimentally verify this fact, one can look at a double beta decay and theorize that there could be a virtual neutrino exchange such that no neutrino is emitted and the other emitted particles would carry their energies. Neutrinoless double-beta decay's rate would be dependent on neutrino mass so it would give better estimates on absolute neutrino mass, It can also provide an explanation for excess matter over antimatter in the current universe and show another instance of lepton number violation. The long half-life of such decay and its small energy distribution are huge experimental challenges, and there are many experiments currently attempting to overcome those hurdles.

