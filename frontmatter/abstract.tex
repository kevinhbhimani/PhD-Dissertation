%The word ÒAbstractÓ should be centered 2? below the top of the page. 
%Skip one line, then center your name followed by the title of the 
%thesis/dissertation. Use as many lines as necessary. Centered below the 
%title include the phrase, in parentheses, Ò(Under the direction of  
%_________)Ó and include the name(s) of the dissertation advisor(s).
%Skip one line and begin the content of the abstract. It should be 
%double-spaced and conform to margin guidelines. An abstract should not 
%exceed 150 words for a thesis and 350 words for a dissertation. The 
%latter is a requirement of both the Graduate School and UMI's 
%Dissertation Abstracts International.
%Because your dissertation abstract will be published, please prepare and 
%proofread it carefully. Print all symbols and foreign words clearly and 
%accurately to avoid errors or delays. Make sure that the title given at 
%the top of the abstract has the same wording as the title shown on your 
%title page. Avoid mathematical formulas, diagrams, and other 
%illustrative materials, and only offer the briefest possible description 
%of your thesis/dissertation and a concise summary of its conclusions. Do 
%not include lengthy explanations and opinions.
%The abstract should bear the lower case Roman number ii (if you did not 
%include a copyright page) or iii (if you include a copyright page).

\begin{center}
\vspace*{52pt}
{\normalfont\textbf{ABSTRACT}}
\vspace{11pt}

\begin{singlespace}
{\khb}: {\thesistitle}\\
(Under the direction of Dr. Julieta Gruszko)
\end{singlespace}
\end{center}

The Large Enriched Germanium Experiment for Neutrinoless double-beta Decay (LEGEND) collaboration plans to search for neutrinoless double-beta ($0\nu\beta\beta$) decay in ${}^{76}$Ge using modular arrays of enriched germanium detectors. Since $0\nu\beta\beta$ candidate events happen at a single site in the germanium detector, pulse shape simulations that model movement of charge carriers in the detector are key to cuts that can reject background from multi-site and surface events. Most events happening in the bulk of the detector, such as gamma-ray events, can be easily simulated by using established models of charge carriers inside Ge. However, surface events such as those caused by alpha incidents on the detector are complex since they generate a large charge cloud, and thus their signal is influenced by effects such as diffusion and self-repulsion. Only the p+ contact and passivated surfaces of the detector are sensitive to alpha events. While these events can be easily rejected using analysis cuts their behavior before cuts, including their energy spectrum and their distribution on the detector surface, is difficult to model. {\tdsim} \footnote{\url{https://github.com/kevinhbhimani/EH-Drift}} is a novel simulation of charge carriers in germanium that incorporates diffusion and self-repulsion to model surface events. We show how such simulations can be sped up using parallel calculations on GPUs, and how they can be used to improve our modeling of surface backgrounds in ${}^{76}$Ge-based $0\nu\beta\beta$ searches.

Traditional simulation rely on a series of first-principles corrections to replicate the detector response, which can be computationally intensive and challenging to generalize across detectors. We present a novel neural network architecture, the Cyclic Positional U-Net (CPU-Net)\footnote{\url{https://github.com/aobol/CPU-Net}}, that performs ad-hoc translations of simulated pulses so that they closely resemble measured detector signals. By using a CycleGAN framework, this Ad-hoc Translation Network (ATN) learns a data-driven mapping between simulated and measured pulses with high fidelity and computational efficiency. We demonstrate, using an HPGe detector, that CPU-Net successfully effectively captures and reproduces critical pulse shape features, enabling more realistic simulations without detector-specific tuning. This generalizable approach can facilitate adaptation to diverse detectors and evolving experimental conditions.

\clearpage
