\chapter{Conclusion}
\label{chap:conclusion}
\section{Summary of Results}
Neutrinos are fundamental to understanding the matter-antimatter asymmetry in the universe. The theory of how neutrinos get their mass suggests that neutrinos could be a Majorana particle. To experimentally verify this fact, one can look at double beta decay and theorize that there could be a virtual neutrino exchange such that no neutrino is emitted and the other emitted particles would carry their energies. HPGe provides a compelling approach to look for neutrino-less double-beta decay since they have an excellent energy resolution and high internal purity. Signal formation is very well understood and can be modeled using simulations. LEGEND collaboration is seeking {\onbb} in Ge-76 using a phased approach and with the goal of investigating the bottom of the inverted hierarchy. Pulse shape simulations that signal formation in the detector help us understand the background model components and simulate the statistical methods used to reject background such as multi-site and surface events. Reliable simulations have been hindered by lack of modeling of surface events and replication of effects of the experiment's electronics response.

Surface backgrounds are the largest components of the LEGEND background model and a significant source of uncertainty due to the lack of proper modeling of charge collection. We developed a new method that can model these challenging backgrounds. {\ehd} extends the capabilities of existing HPGe detector simulations by incorporating detailed processes such as diffusion, self-repulsion, surface charge, and reduced surface drift. Its ability to model large charge cloud drifts and surface drifts makes it a valuable tool for understanding detector response to surface events and improving background modeling in experiments. Using GPU computing, we can significantly accelerate the simulation, allowing for a large-scale background modeling study. Using activeness maps created using {\ehd} we were able to explain the anomalous behaviorism of previous alpha test stands. We also show that maps can be used to calculate the efficacy of collecting {\onbb} events and to create the background spectrum for alpha and beta events.

The readout electronics can influence the readout signal of the HPGe detectors. It is quite challenging to use first-principles corrections to fully account for the electronic response. We developed a CycleGAN network that learns to translate simulated pulse shapes into detector-like signals without explicitly modeling any of the electronics. The model combines a positional U-Net generator with an RNN-based discriminator. This model trains on arbitrarily collected and unpaired data sets of simulated and detector pulses. CPU-Net successfully translates simulated pulses to outputs indistinguishable from actual detector pulses by applying corrections according to proper detector physics.  We showed that the CPU-Net correctly reproduces the distribution of two critical pulse-shaped reconstruction parameters.

Together, the two methods improve the accuracy of pulse shape simulations. This will enable a large-scale simulation study of LEGEND backgrounds, test new pulse shape discrimination strategies, and adapt simulations to different detector configurations.  

\section{Future Direction}

\subsection{Validation of Results}
To validate the results of {\ehd}, a direct comparison of the surface waveform with {\ehd} output can be performed. This will help validate the results while providing the magnitude of range for tuneable parameters such as surface to bulk surface charge, drift velocity, passivated surface depth, etc. A surface scanning cryostat is currently being commissioned at UNC and will provide the data needed to validate the simulations. The scanner aims to perform multidetector alpha measurements with a $^{241}$ Am source and a monoenergetic beta study with a $^{137}$Cs source. Waveforms from the scanner can be used to validate the simulations as the energy and location of events are known from the source type and location. The ability of scanner to perform rapid scanning of different detectors will help us to determine the detector-to-detector variation in the surface effects. This will also help us to understand how these values depend on event location, energy deposition, alpha incidence angle, detector temperature, and dielectric material (such as LAr).

The Compton scanner in TUM provides an opportunity to validate the CPU-Net results \cite{Abt_2022odr}. The scanner operates by irradiating the detector with gamma rays and then collecting the scattered gamma rays with pixilated cameras. This enables creating a library of data waveforms with known position.
We can use this paired dataset to train and validate CPU-net as we can now perform a direct comparison of ATN output with data waveforms.

\subsection{Improving {\ehd} Activeness Maps}
Until now, we have been using cubic interpolation to build activeness maps in {\ehd}. Although this method is a good approximation, it does not always capture the sudden changes in activeness near the passivated surface. This is because the charge-collecting efficiency varies much more rapidly in the near-surface areas where surface effects dominate than in the bulk of the detector. Thus, the cubic interpolated tend to have sharp edges close to the surface. One way to address this is to adopt interpolation that accounts for local gradients or discontinuities in the activeness function. Techniques like radial basis function interpolation could be used to refine the interpolation in high-variation regions while maintaining a coarser interpolation in the bulk. Another possibility is to perform a study on the activeness to determine the optimal sampling of the detector. The electric field near the surface with detector type, energy, and surface charge can give a relation for how the activeness is going to change, and this can be used to sample the detector optimally.


\subsection{Beyond Standard Model Searches}
The {\ehd} also have an important application in low energy studies. Enriched detectors in {\Lthou} have limited cosmogenic exposure on the surface and a very low $^{39}$Ar background event rate, resulting in ultra-low backgrounds at low energy. This enables highly sensitive searches for other physics Beyond the Standard Model (BSM) such as signatures of light weakly interacting massive particles (WIMP) and bosonic dark matter (DM). Low-energy BSM physics searches would require an accurate model of the passivated surface. This is because most background events at low energy happen on the surface, and thus a passivated surface model would help reject them accurately. A model for these passivated surface interactions would directly improve the low E sensitivity in {\Lthou}. BSM effect can alter the shape of $2\nu\beta\beta$, and studying the distribution can provide information about many BSM processes. {\ehd} will play a crucial role in $2\nu\beta\beta$ spectral distortions, since such studies would require detailed modeling of the background of $^{39}$ Ar beta, a passivated surface effect.


\subsection{Future for Pulse Shape Simulations}

The future development in pulse shape simulations for Germanium detectors will be centered around {\ssd} . Although {\ehd} successfully captures diffusion, self-repulsion, and surface charge effects, it also makes several assumptions in 2-D simulations of a 3-D problem that {\ssd} can, in principle, avoid. The fully three-dimensional approach of {\ssd} provides a more complete picture of electric fields and charge drift, and it also provides the ability to model the effects of the surroundings of a detector. The programmer is written in \texttt{Julia} which provides a natural way to parallelize the calculations.

The techniques developed in {\ehd} such as GPU-based drift and diffusion and surface drift can be used to improve {\ssd}. Although {\ssd} already uses GPU for field calculations, performing drift and diffusion on GPUs will enable simulating large charge cloud surface events much faster in {\ssd}. In {\ssd} the number of charges used is an important input. The higher the number of charges, the better the results are, but after a certain number there is a diminishing of results in improvement. A study using comparing {\ehd} results and {\ssd} can help find this optimal point to have accurate simulations while optimizing runtime. We performed some initial {\ssd} simulations with surface drift and the results looked promising.

\subsection{Passivated Surface Background Modeling for LEGEND}
Ultimately we want to create an accurate background model component for the passivated surface event. Using Geant4 simulations, we can estimate the location and energy of alphas deposited on the surface from varying background sources. Using an activeness map, we would then estimate the energy spectrum for those depositions. Alternatively if the {\ssd} provide a more optimal method of simulations we can simulate each event without interpolating the maps themselves. Thus, this would enable us to create a background model that accurately describes the behavior of the passivated surface events. Knowing the fraction of those events appearing in the {\onbb} ROI would also help us reduce the uncertainty in the alpha background contribution in the LEGEND experiment.

\subsection{ A Challenges in Cycle-GAN based Model}
The CPU-Net was quite effective at translating simulated waveforms to data-like pulses, while matching the current amplitude and drift time distribution. However, the CycleGAN training can be quite unstable and requires fine-tuning of the several hyperparameters to get the right balance. This can be seen in figure \ref{ch8_fig_tail_slope_sim}, where the distribution of data and ANT output waveform do not match. We also noticed that the noise added by CPU-Net is not always consistent with what we see in the data. 

To test both effects, we added pole zero decay and pink noise to the simulated waveform. This gave us a direct one-to-one data set for validation. We trained CPU-Net with simulations plus electronics effects as data. After training we observed that while the ATN output waveform is active aligned with the data waveform better, there were still a significant difference between them. This shows that the model learns the electronics response to an extent to fool the discriminator, but not necessary to a high precision. 

\subsection{CPU-Net Model Improvements}
One way to improve the model is to incorporate distribution-based loss metrics, such as the Wasserstein distance, into loss optimization. This can be done for known parameters such as the tail slope, current amplitude, and drift time. This forces the model to learn the correct distribution during training. We would not know how the model works for unknown electronics parameters; however, proving the known parameters can give the model a first guess to improve upon. Combining the Wasserstein metrics with cycle consistency and identity consistency could help the generator learn the features better.Input simulations can also be improved using {\ssd} instead of siggen. This will provide better signal modeling and initial approximation for drift time and current amplitude distributions.

Another approach would be to replace CycleGAN with an advanced generative model such as the diffusion model \cite{2020arXiv200611239H} or the Transformer model \cite{Transformer}. The diffusion model gradually adds Gaussian noise to the data pulse in a Markov chain. Then it trains a network to reverse the chain. This allows for finer control over the learned distribution. The advantage of the method is that we can control the noise being encoded. Thus, using known waveform shape parameter priors, such as tail slop, current amplitude, and drift times, we can guide the model to learn the correct translations of simulated pulses.