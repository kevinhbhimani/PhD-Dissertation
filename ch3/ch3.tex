\chapter{Surface Event Modeling and Pulse Shape Simulations}

\section{Introduction}

The events on passivated surfaces create a unique modeling challenge in HPGe detectors. These interactions can be best understood with alphas, since they deposit energy over short distances on the surface, creating large charge clouds with complex drift and collection processes, making it quite challenging to understand. We therefore study the events in terms of alpha interactions, but the same techniques can be applied to simulations of other passivated surface backgrounds such as beta and low-energy gamma interactions. This chapter begins by reviewing alpha backgrounds, then discusses the challenges in modeling surface backgrounds, and introduces a new model to address these challenges.


\section{Properties of Alpha Interactions}
Alpha particles are composed of two protons and two neutrons tightly bound together. They are identical to ionized Helium ions but carry a charge of $+2$ e. The energy of alpha particles is correlated with the half-life of the parent isotope, such that the ones with the highest energies are from parent atoms with the shortest half-lives. The energies of the alpha particles are usually between about $4$ and $6$ MeV. For a given material, the linear stopping power S is defined as $-\frac{dE}{dx}$. The energy loss of charged particles in a material is then given by the Bethe formula:

\begin{equation}\label{bethe_formula}
    -\frac{dE}{dx} = \frac{4\pi e^4z^2}{m_0\nu^2}NB
\end{equation}

such that

\begin{equation}\label{bethe_B}
    \text{B}=Z\left[ \text{ln}\frac{2m_0\nu^2}{I}-\text{ln}\right(1-\frac{\nu^2}{c^2}\left)-\frac{\nu^2}{c^2}\right]
\end{equation}

Here $\nu$ and $ze$ are the charges of the given particle, N and Z are the number density and atomic number of the absorber atoms, and m$_0$ and e are the rest mass and charge of the electron, respectively. Parameter I represents the average excitation and ionization potential of the absorber and is normally determined experimentally. 

Eq. \ref{bethe_formula} suggests that the energy loss for non-relativistic particles is proportional to $1/\nu^2$, or the kinetic energy. Hence, the slower the velocity, the longer the particles spend in the vicinity of electrons of the material's atoms, and the higher the energy loss. Eq. \ref{bethe_formula} is proportional to z$^2$ for a constant velocity. Thus, the heavier particles will have more energy loss for a given velocity. This results in alphas having a very low penetration depth, usually around $10$ micrometers.

The energy loss of the alpha particles can be best understood by plotting the specific energy loss along the track. The plot known as the Bragg curve is shown in Figure \ref{bragg_curve_fig}. During most of the track, the energy loss increases roughly as 1/E, as predicted by Eq.\ref{bethe_formula}. As the particles slow down, electron pickup reduces the charge and the curve falls off. As a result, alpha deposits a large amount of energy in a very localized area. For Germanium, this penetration depth of alpha particles is about $17.6$ - $20.0$ microns \cite{knoll_2010}. The n+ electrode is too thick for the alphas to penetrate and deposit energy. The p+ electrode in LEGEND detectors does have alpha events, but the charge collection is well understood, as the events are collected immediately and thus can be rejected easily using PSD techniques. Between these contacts is a thin insulating passivated surface, where alphas can penetrate the active volume of the detector, and the charge collection is more complex. The modeling technique developed here is focused on this surface.

\begin{figure}
\centering
\includegraphics[width=0.5\linewidth]{ch3/figs/bragg_curve.png}
\caption{The specific energy loss in a material for an alpha particle with several MeV initial energy. Plots are shown for a single alpha particle track and for the average behavior of a parallel beam of alpha particles. Alphas loss most of their energies in a small region. Picture from \cite{knoll_2010}}. 
\label{bragg_curve_fig}
\end{figure}

\section{Alpha Background Observations}
\label{ch3_sec_alpha}
The alpha background can vary between different detector environments and experiments. {\MJD} and the {\Gerda} experiment had different spectral shapes for alpha backgrounds. In {\MJD}, alpha events degraded more substantially in {\onbb} ROI, while in \Gerda, prominent peaks of the $^{210}$ Po and the $^{226}$ Ra chains were identified above $5$ MeV, along with lower energy tails of partially collected surface alphas. Factors such as the cryostat environment, detector geometry, and material purity can contribute to the alpha background at Q$_{\beta\beta}$.

This can also be seen in dedicated test stands that were developed to study alpha backgrounds. Two scanning systems were developed to study alpha interactions in HPGe detectors: TUBE \cite{TUBE_paper} and GALATEA \cite{galatea_paper}. These scanners yielded conflicting dependencies of the energy loss as a function of radial position. TUBE reported that energy degraded inversely with radius, whereas GALATEA reported the opposite trend. Figure \ref{fig:dcr_waveform} shows a known alpha waveform in the TUBE data. The alpha waveform, being close to the surface, has a much sharper rise than the bulk waveform. However, the tail of the alpha waveform has an upward slope towards the end. This is due to slow collections of charges on the passivated surface. We call this effect the Delayed Charge Recovery Effect (DCR) \cite{Gruszko:2017kfx}. This feature and others like it are used to tag and reject alphas. Such cuts have been highly effective, but it is hard to know which events are not tagged without modeling the signal.

Figure \ref{ch3_fig_L200_surface_background} shows the {\Ltwo} physics spectrum. Muon and multiplicity cuts remove backgrounds such as muons and gamma rays, and the events left are primarily surface events shown by the unfilled histogram. As shown in green, Pulse Shape Discrimination (PSD) cuts are highly effective in removing them, but the efficiency of those cuts cannot be determined due to lack of model for surface events. 

\begin{figure}[!htb]
\centering
\includegraphics[width=0.9\linewidth]{ch3/figs/dcr_waveform.pdf}
\caption{An alpha waveform (red) compared to a bulk waveform (blue) in the MJD PPC detector. The DCR effect results in a slow charge collection toward the end as shown by the insert plot in the box.\cite{TUBE_paper}}
\label{fig:dcr_waveform}
\end{figure}

\begin{figure}[!htb]
\centering
  \includegraphics[width=0.99\linewidth]{ch3/figs/l200-phy-spectrum-psd.png}
  \caption{{\Ltwo} physics spectrum. The events above 3000 keV after the muon and multiplicity cut are primarily alphas. These events can be effectively remove using PSD cuts as shown in green. Credit: LEGEND collaboration}
\label{ch3_fig_L200_surface_background}
\end{figure}


\section{Challenges in Modeling Surface Events}
An alpha particle deposits an energy of approximately $4$ - $6$ MeV with a penetration depth of approximately $17.6$ - $20.0$ microns on the passivated surface. This excites a lot of charge carriers and produces a dense charge cloud. In such cases, effects such as diffusion of the charges and self-repulsion among charges become significant. These effects could also push the charge onto the passivated surface. 

The charges on the passivated surface move at slower speeds than in the detector bulk. This is mainly due to additional scattering mechanisms and modified band structure on the surface, described in \cite{MULLOWNEY201233}. In bulk, the hole and electron transport velocity can be approximated using a simple warped-band approximation. However, on the passivated surface, the band structure is altered by electrical fields normal to that on the surface. This leads to a higher probability of scattering of charges and surface-related effects such as roughness scattering.  Simulations in Ref. \cite{MULLOWNEY201233} estimate that the drift speed on the surface is 40 to 50 slower than in the bulk. 

Alpha events create a highly localized, dense charge cloud close to the surface. Because of the charge that ends up on the surface drifting slower than the bulk, the charge could get stretched into a non-spherical cloud. Thus, spherical charge cloud approximation cannot be used for such events.

%This charge cloud is difficult to model using existing simulations like {\siggen} as it uses a spherical charge approximation.


\section{Surface Charge Effect}
As they are operated, the HPGe detectors can accumulate static charge on the passivated surface, which affects the charge collection for surface events. The sign and magnitude of the accumulated charges can vary depending on the operating conditions. Figure \ref{ch3_fig_surface_field_sc0} shows how the presence of a surface charge alters the electric field close to the passivated surface. When there is no surface charge (top), the field lines are parallel to the surface. The presence of a negative surface charge (middle) attracts the field lines towards the surface. The positive surface charge (bottom) repels the field lines from the surface. The negative surface charge can pull the holes to the surface, where they can drift slowly. Similarly, positive surface charges can pull the electrons to the surface.

\begin{figure}
\centering
%[trim={left bottom right top},clip]
\includegraphics[trim={1.5cm 3.5cm 0.2cm 4.6cm},clip,width=0.99\linewidth]{ch3/figs/elect_field_lines_surface_ponama_1_sc_0.pdf}
\includegraphics[trim={1.5cm 3.5cm 0.2cm 4.6cm},clip,width=0.99\linewidth]{ch3/figs/elect_field_lines_surface_ponama_1_sc_-0.5.pdf}
\includegraphics[trim={1.5cm 3.5cm 0.2cm 4.6cm},clip,width=0.99\linewidth]{ch3/figs/elect_field_lines_surface_ponama_1_sc_0.5.pdf}
\caption{Electric field magnitude and field lines in a region near the passivated surface of a {\ponama} PPC detector with different surface charges. Geometry information for the detector can be found in Table \ref{ch5_5_tab_ponama1_geometry}. Charges on the passivated surface can pull electrons and holes to the surface and cause energy loss and slow charge collection. Top figure has no surface charge and the field lines are parallel to surface. Middle figure has a surface charge of $-0.5 \times 10^{10} \frac{e}{cm}$ which pulls the filed lines to the surface. Bottom figure has surface charge of $0.5 \times 10^{10} \frac{e}{cm}$ that repels the filed lines form the surface.}
\label{ch3_fig_surface_field_sc0}
\end{figure}

Since the surface charge changes the overall electric field of the detector, it could also alter the depletion voltage for the detector. Figure \ref{ch3_fig_deplection_sc} shows how the depletion voltage of a LEGEND PPC detector changes with surface charge. Typically, the detector's operational voltage is higher than the depletion voltage, but if the surface charges are not properly accounted for, a detector could become undepleted after being deployed in the experiment.

\begin{figure}[!htb]
\centering
\includegraphics[trim={1cm 0.4cm 1cm 1.75cm},clip,width=0.99\linewidth]{ch3/figs/deplep_sc.pdf}
 \caption{Effect of surface charge on depletion voltage for LEGEND PPC detector P00661C. Depletion voltages calculated using {\siggen} software. Surface charge can alter the depletion voltage and potentially cause a detector to become undepleted after being deployment.
}
\label{ch3_fig_deplection_sc}
  \end{figure}

Pulse shape simulations can help model these effects accurately and build a background model for surface events. A simulation to model surface events should allow for a non-spherical charge cloud while incorporating surface drift, diffusion, and self-repulsion. They should also properly account for surface charge effects. Next, we look at the current pulse-shape simulation frameworks that are being used by LEGEND.

\section{Current Status in Pulse Shape Simulations}

Signal formation in HPGe detectors is modeled using the Shockley-Ramo theorem. This enables the building of accurate simulations that can model the waveforms. The LEGEND Collaboration uses two HPGe signal modeling software packages, both developed in part by members of the Collaboration: {\siggen} and \texttt{SSD} simulations. 

{\siggen} is a C-based program developed by David Radford at Oak Ridge National Laboratory \cite{siggen_paper}. It consists of two components: \texttt{fieldgen} and {\siggen}. \texttt{fieldgen} is used to calculate the electric potential and weighting potential of point-contact detectors in two-dimensional. {\siggen} then uses the weighting potential calculated from \texttt{fieldgen} to generate signals for a charge originating at a given point in the detector using the Shockley–Ramo theorem. \texttt{fieldgen} simulations can be used to calculate the depletion voltage, the volume of the depletion region, and the capacitance of the detector. Diffusion in {\siggen} is approximated using a Gaussian convolution, and there is no mechanism to model nonspherical charge clouds since {\siggen} uses point charges to represent the entire charge cloud. {\siggen} simulation has been crucial in the design and manufacturing of the LEGEND detector, as it allows precise modeling of electric fields and projected PSD performance.


% A simulated event in {\siggen} is shown in Figure \ref{fig:{\siggen}_1d}.

% \begin{figure}
% \centering
% \includegraphics[width=0.49\linewidth]{ch3/figs/drift_path.png}
% \includegraphics[width=0.49\linewidth]{ch3/figs/samp_wav_comp.png}
% \caption{A simulated event showing the path of holes and electrons and the corresponding waveform}.
% \label{fig:{\siggen}_1d}
% \end{figure}

SolidStateDetectors.jl (\texttt{SSD}) is a software package developed by the Abt group at the Max Planck Institute in Munich \cite{Abt:2021SSD}. It is written in Julia and can perform calculations in full 3-dimensions. \texttt{SSD} can calculate electric fields and potentials outside the detectors. The \texttt{SSD} enables full 3-D diffusion and models self-repulsion in the signal. It does this by using a limited number of charges, each of which represents many real charges. Charges are individually tracked using their position and velocity. The electric field calculation can be performed on GPUs. 


% SSD provides a complete 3-D simulation of the charges in Germanium along, but the computational run-time quickly scales with the number of particles used. 


% Alpha particles typically create a large charge cloud with a number of particles that would be difficult to model with the \texttt{SSD}.

% Somewhere here you need to include the caveat that it does this using a limited number of charges, each of which represents many real charges. You sort of hint at it later, but I'm pretty sure actually trying to simulate every charge would crash the program, not just increase the computation time.

% In bulk, spherical charge cloud model works very well, but the charge cloud produced by alpha is not necessarily spherical due to the charge trapping and re-release effect on the passivated surface. A simulation to model alpha should allow for a nonspherical charge cloud while incorporating surface drift, diffusion, and self-repulsion.

% \begin{figure}
% \centering
% \includegraphics[width=0.49\linewidth]{ch3/figs/\texttt{SSD}_e.png}
% \includegraphics[width=0.49\linewidth]{ch3/figs/\texttt{SSD}_path.png}
% \caption{Simulated Electric field and charge paths in \texttt{SSD} simulations. It allows for calculations outside the detectors and in full 3 dimensions. \cite{\texttt{SSD}_web}.}
% \label{fig:\texttt{SSD}_plots}
% \end{figure}


\section{{\ehd}}
{\ehd} is a newly developed method to simulate surface events while directly simulating diffusion and self-repulsion. It is a specialized model designed to simulate passivated surface interactions. The model uses 2-D approximations to optimize run-time while adjusting for 3-D effects. The original software version was developed in C by David Radford and built as an extension to {\siggen}. As charges drift through the detectors, the software keeps track of charge densities at a pixel-by-pixel level, which allows for nonspherical charge clouds. The charges that end up on the surface have their velocities reduced by a predetermined factor. {\ehd} also enables simulation of the effects of surface charges. The simulation is performed in the r and z directions while assuming $\phi$ symmetry; therefore, the charge cloud is simulated by a ring of charges, and the calculations account for the 3-D effects using analytic approximations. The charges that end up on the surface have their velocities reduced by a predetermined factor. This approach does not take into account the varying mobilities as a function of the crystal axis because this is a small effect for events dominated by surface drift.

% Inspired by fluid mechanics
To solve for the partial differential equations that describe charge movements, {\ehd} takes the Lagrangian splitting approach. This means that each step, such as diffusion, charge drift, etc., is solved independently, and then the net effect is calculated by combining them. Figure \ref{fig:ehd_flowchart} shows how the {\ehd} program works. 

During the initial setup, the detector is divided into a grid. Each point on a gird has a value for the potential, weighting potential, electron- and hole densities, and impurity. The boundary conditions are then set according to the detector input geometry, surface charge, and bias voltage. Then electric potential and weighting potential are calculated using an over-relaxation algorithm. The capacitance and depletion are estimated to understand the detector behavior. The capacitance is calculated by relating two equations for the energy stored in an electric field:

\begin{equation}
    C = \frac{\epsilon \int E^2 \, dV}{V^2}
\end{equation}

The depletion voltage is estimated by finding the minimum bias required to fully deplete the detector. Then the software checks to find local minima in the potential to help find the pinch-off depletion region. The initial charge densities are distributed at a given location based on the impact energy, usually over two adjacent grid points. The initial densities are determined using the following:

% \begin{equation}
%     \rho_H = \rho_E= \frac{\text{ E}\times 10^7 }{1000 \times 0.003 \times \text{grid}^3}
% \end{equation}

\begin{equation}
\rho_E \Bigl(10^{10} \,\frac{\mathrm{pairs}}{\mathrm{cm^3}}\Bigr) = \rho_H  \;=\;
\frac{ E\bigl(\mathrm{keV}\bigr)\,\times\,\bigl(10^{3}\,\mathrm{cm}^{3} = 1\,\mathrm{mm}^3\bigr)\,\times\,10^{-10} }{ \bigl(1000\,\mathrm{keV} = 1\,\mathrm{MeV}\bigr) \,\times\,\bigl(0.003\,\mathrm{MeV/pair}\bigr)\,\times\,\bigl[\text{grid}\,(\mathrm{mm})\bigr]^3 }
\label{eq:density_with_units}
\end{equation}



where $\rho$ represents the density of the charge pair in units of \(10^{10}/\text{cm}^3\). The energy deposited E is the energy deposited in keV, which is first converted to MeV by dividing by 1000.0. The factor of \(10^7\) is to get the final unit correct. The approximate energy required to generate an electron-hole pair in Germanium is $0.003$ MeV. Finally, dividing by \(\text{grid}^3\) in mm normalizes the number of charge pairs generated with the volume of the grid to get the density in units of $10^{10} \,\frac{\mathrm{pairs}}{\mathrm{cm^3}}$. The densities for both holes and electrons are deposited equally in points (z,r) and (z+1,r). The density is then multiplied by a calibration factor that is tuned such that the 2-D simulation of the self-repulsion effect matches the 3-D model. The calibration factor we found was 20.

\begin{figure}[!htb]
\centering
%[trim={left bottom right top},clip]
\includegraphics[width=0.99\linewidth,trim={2pc 10pc 1.5pc 9pc},clip]{ch3/figs/ehd_flowchart.pdf}
\caption{A flow chat illustration the {\ehd} steps. The initial steps to calculate intrinsic parameters of the detector (e.g. weighting potential, capacitance, etc.) are performed once. Then subsequent loops step through each time step and each grid point to track the movement of charge clouds.}
\label{fig:ehd_flowchart}
\end{figure}

After the initial setup, the program steps through a series of time steps. The charges are allowed to diffuse and then drift in the electric field. The calculation of the electric field is discussed in Section \ref{ch4_sec_relax_algo}. In the following sections, we discuss the rest of the critical components of {\ehd}.

\subsection{Diffusion}
Diffusion arises from the random thermal motion of electrons and holes, which causes them to spread out from regions of high concentration into regions of lower concentration. In thermal equilibrium, the diffusion coefficient $D$ and the mobility $\mu$
are related by the Einstein relation:
\begin{equation}
D = \mu \frac{k_B T}{q},
\end{equation}
where $k_B$ is Boltzmann's constant, $T$ is the absolute temperature, and $q$ is the elementary charge. Thus, if the mobility $\mu$ is known or has been fit to experimental data, $D$ can be determined for a specified temperature. In germanium at cryogenic temperatures (e.g. $\sim 77$, K), the mobilities are
sufficiently large that diffusion can still have a noticeable effect on the final spread of a charge cloud, although it is smaller than under room temperature conditions.

In {\ehd} the diffusion is simulated by redistributing the charge among neighboring grid cells in the $(r)$ and $(z)$ directions using the diffusion coefficients. During each time step $\Delta t$, a fraction of the carriers in the cell $(z,r)$ diffuse to adjacent cells $(z\pm1,\,r\pm1)$ according to two quantities: $\delta_z$ and $\delta_r$. These fractions are computed from an approximate diffusion parameter $f$ and the local drift velocity magnitudes $\lvert \mathbf{v_z} \rvert$, $\lvert \mathbf{v_r} \rvert$ as follows:
\begin{align}
   \delta_z &\;=\; \frac{\Delta x \;\,v_z \;f}{E_z}\label{eq:deltaez}\\
   \delta_r &\;=\; \frac{\Delta x \;\,v_r \;f}{E_r} \label{eq:deltaer}
\end{align}
where $\Delta x$ is the grid size, $v_{z}$ and $v_{r}$ are the drift velocities, $E_{z}$ and $E_{r}$ are the local field components, and $f$ incorporates the diffusion coefficient $D$ and corrections for grid sizes. If $E_z$ or $E_r$ is below $1\,\mathrm{V/cm}$, the program sets $\delta_z = 0$ or $\delta_r=0$ assuming that the low field regions produce negligible drift velocity.

\subsubsection*{Volume Correction}

In cylindrical $(r,z)$ coordinates, each radial ring has a different physical area. The simulation tracks these scaling factors with arrays:
\[
\texttt{s1}[r] = 1 + \frac{0.5}{r-1}, 
\qquad
\texttt{s2}[r] = 1 - \frac{0.5}{r-1},
\] 

For r=1 and r=0, the code uses a special fixed weighting factor to handle the geometry. These scalars adjust the amount of charge diffuses into or out of the neighboring ring. Specifically, if $\delta_r$ is the fraction of carriers leaving cell $(r)$ in
the $+r$ direction, the actual increase in the neighbor cell
$(r+1)$ is multiplied by $\texttt{s1}[r]$, while the old cell is
reduced by the same fraction multiplied by $\texttt{s1}[r]$ to
balance volume differences. Similarly, diffusion to cell $(r-1)$ 
uses $\texttt{s2}[r]$. The same volume correction is used in the field calculation.

\subsubsection*{Diffusion Density Update}

After computing the diffusion fractions $\delta_z$ and $\delta_r$, the code subtracts those amounts from the point's density and adds them to the four neighbors:
\begin{align}
  \rho^{\mathrm{new}}(z,\,r+1) &\;\mathrel{+}= \;\rho^{\mathrm{old}}(z,\,r)\times\delta_r \times s_1(r) \times \frac{r-1}{r} \label{ch3:eq:diffusion_update_1} \\
  \rho^{\mathrm{new}}(z-1,\,r) &\;\mathrel{+}= \;\rho^{\mathrm{old}}(z,\,r)\times\delta_z \label{ch3:eq:diffusion_update_2} \\
  \rho^{\mathrm{new}}(z+1,\,r) &\;\mathrel{+}= \;\rho^{\mathrm{old}}(z,\,r)\times\delta_z \label{ch3:eq:diffusion_update_3} \\
  \rho^{\mathrm{new}}(z,\,r-1) &\;\mathrel{+}= \;\rho^{\mathrm{old}}(z,\,r) \times \delta_r \times s_2(r) \times\frac{r-1}{r-2} \label{ch3:eq:diffusion_update_4}.
\end{align}
 The offsets $r$ and $r-2$ are due to a one-based indexing in the code.
 
\subsection{Drift in Electric Field}
Germanium has a diamond-cubic lattice structure. The crystallographic basis axes are in the $<h100>$, $<h110>$, and $<h111>$ directions. Charges in the three basis axes travel at different velocities. Mobility relates the velocity of the charges to the electric field. At low field, the relationship is Ohmic:
\begin{equation}
\vec{v} = \mu_0 \vec{E}
\end{equation}
At higher field, the scattering with the crystal lattice results in a nonlinear relationship. This scattering causes the velocity to saturate at higher values. The relationship in a higher field can be modeled using \cite{Caughey_1448053}:

\begin{equation}
v(E) = \frac{\mu_0 E}{(1 + (E/E_0)^\beta)^{1/\beta}}
\end{equation}

In {\ehd}, we calculate the electric field in the r and z directions at each point of the grid. We use values from \cite{OMAR19871351}, which gives $v(E)$ from the local electric field $E$. We use a piecewise linear interpolation to find $v(E)$. To do this, we define an array of electric field points, $\texttt{drift\_E}$, which partitions the field range into intervals. For each interval $[E_i, E_{i+1}]$, we store two quantities: a drift offset, corresponding to the drift velocity at $E = E_i$, and a drift slope, which is the rate of change of the drift velocity with respect to the electric field in that interval. The drift velocity is then
computed by:
\[
v(E) \;=\; \text{drift offset}[i]
\;+\; \text{drift slope}[i] \,\bigl( E - E_i \bigr),
\]
with $E_i \le E < E_{i+1}$. The Figure \ref{ch3_fig_dv_vs_e} simulates the relation for the Drift Velocity versus Electric field. The drift velocity is then multiplied by the time step to find where the charges will drift, and then the charges are moved to the new location. For simplicity, we only used the $<100>$ direction velocities in the current model. As shown in Figure \ref{ch3_fig_dv_vs_e}, for a given electric field magnitude, the variation in mobility with the crystal axis is expected to be within the same order of magnitude, while the surface drift drift could be up to $1000$ times slower.

\begin{figure}[!htb]
    %[trim={left bottom right top},clip]
    \includegraphics[trim={0cm 0 0cm 0},clip,width=0.99\linewidth]{ch3/figs/ehd_dv_e.pdf}
    \caption{Relation of drift velocity verses electric field in {\ehd}. The points shows the experimental values are adapted from \cite{OMAR19871351}. To show how {\ehd} estimates the values we create 500 samples between 0 and 5000 V/cm and performed the piecewise linear interpolation between the points. Only $<100>$ direction velocities are used in the current model.}
    \label{ch3_fig_dv_vs_e}
\end{figure}

\subsubsection*{Fraction Splitting}\label{ch3:sec:frac_split}
We use a splitting approach to move charges to the new cell. Since we are using a grid, we want to account for the discretization by splitting the charges between two grid points. Suppose that after a time step $\Delta t$, the density in the grid cell $(z,r)$ moves to a new position with integer indices $(k,i)$. We define fractional parts $f_z$ and $f_r$ as:
%
\begin{align}
k \;=\; z + \lfloor \Delta z \rfloor,\quad
f_z \;=\; \Delta z - \lfloor \Delta z \rfloor,\\
i \;=\; r + \lfloor \Delta r \rfloor,\quad
f_r \;=\; \Delta r - \lfloor \Delta r \rfloor,
\end{align}


The charges are split using these fractions:
\begin{align}
&\text{fraction in }(k, i)   \;=\; f_{r} \times f_{z} \\
&\text{fraction in }(k, i+1) \;=\; (1 - f_{r}) \times f_{z}\\
&\text{fraction in }(k+1, i) \;=\; f_{r} \times (1 - f_{z})\\
&\text{fraction in }(k+1, i+1)\;=\; (1 - f_{r}) \times (1 - f_{z}).
\label{eq:bilinear-fractions}
\end{align}

Finally, the densities are updated using:
\begin{align}
\rho^{\mathrm{new}}(k,i)   &\,\mathrel{+}=\, \rho^{\mathrm{old}}(z,r)\times \bigl[f_{r}\times f_{z}\bigr] \times \text{G}_{r,z} \label{ch3:eq:den_update_1} \\
\rho^{\mathrm{new}}(k,i+1) &\,\mathrel{+}=\, \rho^{\mathrm{old}}(z,r)\times \bigl[(1 - f_{r})\,f_{z}\bigr] \times \text{H}_{r,z} \label{ch3:eq:den_update_2} \\
\rho^{\mathrm{new}}(k+1,i) &\,\mathrel{+}=\, \rho^{\mathrm{old}}(z,r)\times \bigl[f_{r}\,(1 - f_{z})\bigr] \times \text{G}_{r,z} \label{ch3:eq:den_update_3} \\
\rho^{\mathrm{new}}(k+1,i+1)&\,\mathrel{+}=\, \rho^{\mathrm{old}}(z,r)\times \bigl[(1 - f_{r})\,(1 - f_{z})\bigr] \times \text{H}_{r,z}. \label{ch3:eq:den_update_4}
\end{align}
\subsection{Geometric Factors}
\label{sec:geom-factor}

In cylindrical coordinates, each cell at index $r$ corresponds to an annular region of approximate circumference $2\pi (r \,\Delta r)$ and thickness $\Delta z$. When the charge moves from cell $(z,r)$ to cell $(k,i)$, we apply the factors:
\[
\text{G}_{r,z} \;\equiv\; \frac{(r-1)}{(i-1)}
\qquad
\text{H}_{r,z} \;\equiv\; \frac{(r-1)}{(i)}
\] 

reflecting the difference in volumes at radii $r$ and $i$.
The offsets $(r-1)$ and $(i-1)$ are for one-based indexing in the code. For points near $r=0$ or $i=0$, the code uses a modified factor that preserves volume scaling. 

% The factors \verb|8*r - 8| or \verb|8*i - 8| are specialized scalings to approximate the annular volume near the central axis or near $i=0$. They ensure that even when $r$ or $i$ equals 0 or 1, the total charge remains consistent with the expected geometry. For example, at the very center ($r=0$), a full ring circumference does not exist, so a smaller effective volume must be used. Although such \verb|8*-8| scalings can appear ad~hoc, they are designed to smoothly transition from the central axis to the first few radial rings, thereby avoiding division by zero while preserving charge.

\subsection{Surface Drift}

When charge carriers drift within the detector, they may encounter the passivated surface, which is handled separately. To model the surface, the lowest grid point is split into two lengths: the length of the surface and the grid minus the length of the surface. We store the charges on the surface in a special row. 

If the height to which the charges drift is less than zero, they are fully added to the surface row. If the charge enters the passivated layer, we split it up using the fraction splitting used in charge drift. Performing the diffusion on the last grid point in Eq. \ref{ch3:eq:diffusion_update_2}, the bottom z-point is considered to be the passivated surface. 


We employ the same methods described above for drifting and diffusing the charges present on the surface, but the drift and diffusion are suppressed by the surface drift velocity factor. This factor, typically about 0.001, is the ratio of the speed on the surface to the bulk. The charges from the surface can drift to the bulk or to other points on the surface. Points on the surface can diffuse to other points on the surface.

\subsection{Courant Number and Adaptive Time Step}

The Courant–Friedrichs–Lewy (CFL) condition is a requirement for numerical solutions of partial differential equations that involve moving particles \cite{cfl_condition}. It is commonly used in fluid dynamics to ensure that particles travel only to the immediate grid points. We introduce the CFL condition in the {\ehd} to ensure that the time step is small enough such that the distance charges drift is not too large during a step. This helps maintain consistency for different input grid sizes. During each simulation time step, we calculate the local Courant number at every point on the grid using the following.

\begin{equation}
C(z,r) = \max \left( \frac{v_r \Delta t}{\Delta x}, \frac{v_z \Delta t}{\Delta x} \right),
\end{equation}

where \( v_r \) and \( v_z \) are the drift velocities in r and z directions, respectively, and  \( \Delta x \) is the grid spacing. The time step is updated using the largest Courant number on the grid :

\begin{equation}
\Delta t = \frac{1}{\max (C(z,r))}.
\end{equation}



\subsection{Impurity Correction}
Surface events create a large number of charge carriers that could impact local impurity and must be corrected to accurately calculate the electric potential. At a given time, the impurity correction is given by:

\begin{equation}
  {\text{I}_{t}}(z, r) = \text{I}_{0}(z, r) +
  \bigl( \rho_h(z, r) - \rho_e(z, r) \bigr) \times \frac{e}{\epsilon} \times \frac{(\Delta x)^2}{2}.
\end{equation}
The $\text{I}_{0}$ is the impurity of the crystal from production. $\rho_h(z,r)$ and $\rho_e(z,r)$ are the hole and electron densities, respectively. $\frac{e}{\epsilon}$ is the conversion factor that relates charge density to the resulting electric field, derived from Gauss’s law in Germanium with $\epsilon = 16\epsilon_0$. $\frac{e}{\epsilon} = 11.310$ in 
charge units $10^{10}\frac{e}{cm^3}$. $(\Delta x)^2$ is the area of the grid which converts density into charge, and then the conversion factor converts it into $10^{10}\frac{e}{cm^3}$ units to match the units of impurity used.

\subsection{Signal Calculation}
In cylindrical coordinates, each grid cell with radius $r$ and height $z$ has an approximate area proportional to $(r - 1)$. Once we read the electron 
density $\rho_e(r,z)$ and hole density $\rho_h(r,z)$ at time $t$, the weighted sums are defined as:
\begin{align}
S_e(t) \;=\; \sum_{z=1}^{L-1} \sum_{r=1}^{R-1}
   \rho_e(r,z)\,\bigl(r-1\bigr)\,\text{wpot}[r-1][z-1],\\
S_h(t) \;=\; \sum_{z=1}^{L-1} \sum_{r=1}^{R-1}
   \rho_h(r,z)\,\bigl(r-1\bigr)\,\text{wpot}[r-1][z-1],
\end{align}
where \(\texttt{wpot}[\,r-1\,][\,z-1\,]\) is the weighting potential at that grid 
point. These $S_e(t)$ and $S_h(t)$ represent the induced signal contributions of electrons and holes, respectively, using the Shockley-Ramo theorem.

The initial unweighted sums are defined as:
\begin{align}
R_{e}(0) \;=\; \sum_{z=1}^{L-1} \sum_{r=1}^{R-1}
   \rho_e(r,z)\,\bigl(r-1\bigr), \quad \\
R_{h}(0) \;=\; \sum_{z=1}^{L-1} \sum_{r=1}^{R-1}
   \rho_h(r,z)\,\bigl(r-1\bigr), \quad
\end{align}
% They keep track of how many electrons or holes remain in the crystal at time $t$ 
% (as opposed to how much they contribute to the signal).  For instance, if 
% $R_{h}(t)$ drops below $R_{h}(0)$, that implies some fraction of holes 
% has reached the collecting electrode or left the active region.

Combining these quantities, the induced signal at time t is computed using:
\begin{equation}
\text{Signal}[\,t\,] 
\;=\;
  \frac{\,S_h(t)\;-\;S_h(0)\,}{\,R_{h}(0)\,}
  \;-\;
  \frac{\,S_e(t)\;-\;S_e(0)\,}{\,R_{e}(0)\,}
\label{eq:net-signal}
\end{equation}
Since the electrons’ contribution would be negative, we subtract it from holes to get the net induced signal. Normalization by $R_{e}(0)$ and $R_{h}(0)$ ensures that the signal is expressed as a fraction of the original total electron/hole count, so it starts near zero at $t=0$ and approaches 1 as all charges arrive at the contacts.

\subsection{Density Snapshots}

The ability to store snapshots of density at each time step provides detailed information about how the charges drift in the detector. Figure \ref{ch3_fig_ehd_path_pd_sc_0} shows a snapshot at time=$80$ns in the {\ehd} for a 5 MeV energy deposition close to the surface without any surface charge. The event started at r=15 mm. The electron drifts towards n$^+$ contact which is located toward the right at $r=34.7$mm, and the hole clouds drift towards the p$^+$ contact towards the right starting at $r=1.5$mm. A 5 MeV energy can create a charge cloud 1.5 mm tall. Charges in such a large charge cloud could drift and diffuse to the surface. The projected density shows the distribution of the charge in the cloud on the x-axis. The head of each signal is the peak, where charges would normally travel in bulk. The tail of the projected density has a peak that is due to the charges that were initially pushed to the surface. The points in between the two peaks are charges that drifted or diffused to the surface in intermediate steps. In this case, the surface drift is set to $1000$ slower than the bulk drift. 

\begin{figure}%[!htb]
    %[trim={left bottom right top},clip]
    \includegraphics[trim={0cm 0 0cm 0},clip,width=0.99\linewidth]{ch3/figs/drift_path_sc=0.0.png}
    \caption{Drift of electron and hole charge clouds in {\ehd}. The projected densities show how the charges are distributed along the radius. The density have two peak, one due to fast moving component in bulk and another due to slow moving component on the passivated surface.}
    \label{ch3_fig_ehd_path_pd_sc_0}
\end{figure}

Figure \ref{ch3_fig_ehd_path_pd_sc_neg_0p3} shows the same event but with a negative charge on the surface. The negative charges alter the field in such a way that holes are pulled onto the surface and electrons are repelled. Thus, the holes component has significant charges attracted to the surface, as shown by their projected densities. Electrons, being repelled from the surface, have no slow-moving component.


\begin{figure}%[!htb]
    %[trim={left bottom right top},clip]
    \includegraphics[trim={0cm 0 0cm 0},clip,width=0.99\linewidth]{ch3/figs/drift_path_sc=-0.3.png}
    \caption{Drift of electron and hole charge clouds in {\ehd} with negative surface charge. The negative surface charge pulls the holes onto the surface which move at a slower speed.}
    \label{ch3_fig_ehd_path_pd_sc_neg_0p3}
\end{figure}

Similarly, Figure \ref{ch3_fig_ehd_path_pd_sc_pos_0p3} shows the same event but with a positive charge on the surface. In this case, the electrons are pulled onto the surface and drift slowly, while the holes are repelled and do not have surface drift. 

\begin{figure}%[!htb]
    %[trim={left bottom right top},clip]
    \includegraphics[trim={0cm 0 0cm 0},clip,width=0.99\linewidth]{ch3/figs/drift_path_sc=0.3.png}
    \caption{Drift of electron and hole charge clouds in {\ehd} with positive surface charge. The positive surface charge pulls the electrons onto the surface which move at a slower speed.}
    \label{ch3_fig_ehd_path_pd_sc_pos_0p3}
\end{figure}


\subsection{Tunable Parameters}
{\ehd} also provides several custom tunable parameters such as surface charge, surface to bulk drift ratio, initial energy, passivated surface depth, etc. that can be used to match the data. Figure \ref{fig:wf_comp} illustrates how the output changes with surface-to-bulk drift and surface charge. The high magnitude of the surface charge means that more charges will be pulled onto the surface and thus a sharp rising part will have a lower magnitude. A faster surface-to-bulk ratio means that the charges on the surface will be collected faster, and therefore the tail shape will be different. Together, these tunable parameters enable fine-tuning of the simulations to match the observed data.

\begin{figure}%[!htb]
    %[trim={left bottom right top},clip]
    \includegraphics[trim={0.1cm 0.3cm 1.3cm 0.3cm},clip,width=0.99\linewidth]{ch3/figs/wf_comp.pdf}
    \caption{A comparison of waveforms generated by {\ehd} for a {\Ltwo} PPC detector based of different tunable parameters surface charge ($\sigma$) and relative surface drift velocity (SD). The events were at r=15 mm z =0.02 mm.}
    \label{fig:wf_comp}
\end{figure}


\subsection{Input and Output}
{\ehd} is compiled using the gcc compiler. It takes in a configuration file the same as that of {\siggen}. The configuration file contains information on the detector geometry, and an example can be found in \cite{ehdrift2024}. In addition to the configuration file, we provide flags that can be passed to the executable. Table \ref{ch3_tab_ehdrift_parameters} shows the input flags that allow one to set multiple parameters related to the event.
\begin{table}[!htb]
    \centering
    \renewcommand{\arraystretch}{1.3} % Adjust row spacing
    \begin{tabular}{|l|p{10cm}|c|}
        \hline
        \textbf{Flag} & \textbf{Description} & \textbf{Example} \\ 
        \hline
        -r & Set the radial position of the event in mm. & 15.00 \\
        \hline
        -z & Set the axial position of the event in mm. & 0.50 \\
        \hline
        -g & Specify the detector name. & P42575A \\
        \hline
        -s & Set the surface charge density in $10^{10} e/\text{cm}^2$. & -0.50 \\
        \hline
        -e & Input the interaction energy in keV. & 5000 \\
        \hline
        -v & Choose whether to write density files (0 = no, 1 = yes). & 1 \\
        \hline
        -f & Choose whether to recalculate the electric field (0 = no, 1 = yes). & 1 \\
        \hline
        -w & Choose whether to save the electric field (0 = no, 1 = yes). & 1 \\
        \hline
        -d & Choose whether to save the depletion surface (0 = no, 1 = yes). & 1 \\
        \hline
        -p & Choose whether to write the weighting potential (0 = no, 1 = yes). & 1 \\
        \hline
        -b & Set the bias voltage in volts. & 3500 \\
        \hline
        -h & Specify the grid size in mm. & 0.0200 \\
        \hline
        -m & Define the passivated surface depth in mm. & 0.10 \\
        \hline
        -c & Set the velocity of surface charges compared to bulk. & 0.75 \\
        \hline
        -a & Input a custom impurity density profile file. & filename.dat \\
        \hline
        -t & Define the total simulation run time in ns. & 16000 \\
        \hline
        -u & Set the frequency of output signal saving in ns. & 16 \\
        \hline
    \end{tabular}
    \caption{Input parameters to EH-Drift}
    \label{tab:ehdrift_parameters}
\end{table}



The simulation output is stored in an HDF5 file, a hierarchical data format optimized for handling large structured datasets. Each simulated event is recorded in the ``event data'' data set as a compound data type containing energy, radius and height positions, surface charge, surface drift velocity factor, and the signal for the event. The data set is dynamically extendable to allow new events to be appended without rewriting existing data. The file includes grid size, passivated surface thickness, self-repulsion flag, and detector name as attributes. Attributes are stored as scalars at the file root level for efficient retrieval without redundancy. This storage is critical for High Performance Computing (HPC) when we simulate thousands of waveforms for multiple detectors.

% To summarize, surface backgrounds are an important component in the LEGEND background model, and are difficult to model. We developed a new method that can model these challenging backgrounds. In the next chapter, we discuss a method to optimize these simulations.

