\begin{center}
\vspace*{52pt}
{\normalfont\textbf{ABSTRACT}}
\vspace{11pt}

\begin{singlespace}
{\authorname}: {\thesistitle}\\
(Under the direction of Dr. Julieta Gruszko)
\end{singlespace}
\end{center}

% The Large Enriched Germanium Experiment for Neutrinoless Double-Beta Decay (LEGEND) collaboration plans to search for neutrinoless double-beta ($0\nu\beta\beta$) decay in ${}^{76}$Ge using modular arrays of enriched germanium detectors. Since $0\nu\beta\beta$ candidate events happen at a single site in the germanium detector, pulse shape simulations that model the movement of charge carriers in the detector are key to cuts that can reject background from multisite and surface events. Most events happening in the bulk of the detector, such as gamma-ray events, can be easily simulated by using established models of charge carriers inside Ge. However, surface events such as those caused by alpha incidents on the detector are complex since they generate a large charge cloud, and thus their signal is influenced by effects such as diffusion and self-repulsion. Only the p+ contact and passivated surfaces of the detector are sensitive to alpha events. Although these events can be easily rejected using analysis cuts, their behavior before cuts, including their energy spectrum and their distribution on the detector surface, is difficult to model. {\ehd} \footnote{\url{https://github.com/kevinhbhimani/{\ehd}}} is a novel simulation of charge carriers in germanium that incorporates diffusion and self-repulsion to model surface events. We show how such simulations can be sped up using parallel calculations on GPUs and how they can be used to improve our modeling of surface backgrounds in ${}^{76}$Ge-based $0\nu\beta\beta$ searches.

% Traditional simulation relies on a series of first-principles corrections to replicate the detector electronics response, which can be computationally intensive and challenging to generalize across detectors. We present a novel neural network architecture, the Cyclic Positional U-Net (CPU-Net)\footnote{\url{https://github.com/aobol/CPU-Net}}, which performs translations of simulated pulses so that they closely resemble measured detector signals. Using a CycleGAN framework, this Ad-hoc Translation Network (ATN) learns a data-driven mapping between simulated and measured pulses with high fidelity and computational efficiency. We demonstrate, using an HPGe detector, that CPU-Net effectively captures and reproduces critical pulse shape features, enabling more realistic simulations without detector-specific tuning. This generalizable approach can facilitate adaptation to diverse detectors and evolving experimental conditions.

The Large Enriched Germanium Experiment for Neutrinoless Double-Beta Decay (LEGEND) collaboration is searching for neutrinoless double-beta ({\ensuremath{0\nu\beta\beta}}) decay in ${}^{76}$Ge using modular arrays of enriched germanium detectors. Pulse shape simulations (PSS) are essential for modeling the analysis techniques used to distinguish signal event and backgrounds from multi-site and surface events, and for generating reliable energy spectra. Together, these capabilities are critical for identifying the sources of background in current experiments and projecting background levels in future searches. Two longstanding obstacles have hindered reliable PSS: incomplete models of surface event interaction and the complexity of replicating the detector’s electronic response. This dissertation addresses both issues using advanced computational techniques.

{\ehd} \footnote{\url{https://github.com/kevinhbhimani/EH-Drift}} is a novel simulation of charge carriers in germanium that incorporates surface drift, diffusion, and self-repulsion to model surface events. {\ehd} accurately simulates the passivated surface events while reproducing the experimental results. Using parallel computing on GPU {\ehd} achieves 62.5x run-time improvement over CPU implementations, making their use for L-200 background modeling computationally feasible.

Cyclic Positional U-Net (CPU-Net)\footnote{\url{https://github.com/aobol/CPU-Net}} is a neural network architecture that performs translations of simulated pulses so that they closely resemble measured detector signals. Using a CycleGAN framework, this Ad-hoc Translation Network learns a data-driven mapping between simulated and measured pulses with high fidelity and computational efficiency. Using an HPGe detector, we show that CPU-Net effectively captures and reproduces critical pulse shape features, allowing more realistic simulations without detector-specific tuning.
\clearpage
