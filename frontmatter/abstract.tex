\begin{center}
\vspace*{52pt}
{\normalfont\textbf{ABSTRACT}}
\vspace{11pt}

\begin{singlespace}
{\authorname}: {\thesistitle}\\
(Under the direction of Julieta Gruszko)
\end{singlespace}
\end{center}

The Large Enriched Germanium Experiment for Neutrinoless Double-Beta Decay (LEGEND) collaboration is searching for neutrinoless double-beta ({\onbb}) decay in ${}^{76}$Ge using modular strings of enriched germanium detectors. Pulse shape simulations (PSS) are essential for modeling the analysis techniques used to distinguish signal events from multi-site and surface event backgrounds, and for generating reliable simulations of energy spectra. Together, these capabilities are critical for identifying the sources of background in current experiments and projecting background levels in future searches. Two longstanding obstacles have hindered reliable PSS: incomplete models of charge collection from surface event interactions and the complexity of replicating the detector’s electronic response. This dissertation addresses both issues using advanced computational techniques.

{\ehd}\footnote{\url{https://github.com/kevinhbhimani/EH-Drift}} is a novel simulation of charge carriers in germanium that incorporates surface drift, diffusion, and self-repulsion to model surface events. {\ehd} accurately simulates the passivated surface events while reproducing the experimental results from Germanium detector test stands. Using parallel computing on GPUs {\ehd} achieves 62.5x run-time improvement over CPU implementations, making their use for L-200 background modeling computationally feasible.

Cyclic Positional U-Net ({\cpunet})\footnote{\url{https://github.com/aobol/CPU-Net}}is a neural network architecture that performs translations of simulated pulses so that they closely resemble measured detector signals. Using a CycleGAN framework, this Ad-hoc Translation Network learns a data-driven mapping between simulated and measured pulses with high fidelity and computational efficiency. Using data from an HPGe detector, we show that {\cpunet} effectively captures and reproduces critical pulse shape features, allowing more realistic simulations without detector-specific tuning.