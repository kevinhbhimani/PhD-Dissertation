\chapter{Conclusion}
\section{Summary of Results}
Neutrinos are fundamental to understanding the origin of the universe. The theory of how neutrinos get their mass suggests that neutrinos could be a Majorana particle. To experimentally verify this fact, one can look at double beta decay and theorize that there could be a virtual neutrino exchange such that no neutrino is emitted and the other emitted particles would carry their energies. HPGe provides a compelling approach to look for neutrino-less double-beta decay. Signal formation is very well understood and can be modeled using simulations.

LEGEND collaboration is seeking {\onbb} in Ge-76 using a phased approach and with the goal of investigating the bottom of the inverted hierarchy. Surface backgrounds are the largest components of the background model and a significant source of uncertainty due to the lack of proper modeling. We developed a new method that can model these challenging backgrounds. {\ehd} extends the capabilities of existing HPGe detector simulations by incorporating detailed physical processes such as diffusion, self-repulsion, surface charge, and reduced surface drift. Its ability to model large charge cloud drifts and surface drifts makes it a valuable tool for understanding detector response to surface events and improving background modeling in experiments. d energy degradation and delayed charge behavior.

The readout electronics can influence the readout signal of the HPGe detectors. It is quite challenging to use first-principles corrections to fully account for the electronic response. We developed a CycleGAN network that learns to translate simulated pulse shapes into detector-like signals without explicitly modeling any of the electronics. This model trains on arbitrarily collected and unpaired data sets of simulated and detector pulses. CPU-Net successfully translates simulated pulses to outputs indistinguishable from actual detector pulses by applying corrections according to proper detector physics. The model combines a positional U-Net generator with an RNN-based discriminator. We showed that the CPU-Net correctly reproduces the distribution of two critical pulse-shaped reconstruction parameters.

\section{Future Direction}
% \subsection{Improving interpolation for {\ehd} maps}
% Until now, we have been using cubic interpolation to interpolate the activeness map. While that is a good approximation, the interpolation is not exactly smooth. This is because the activeness close to the surface changes more drastically than that of those deeper in the bulk.  

% \subsection{Future for Pulse Shape Simulations}
% - SSD is the future of PSS simulations. In {\ehd} we make many approximations which we do not have to in SSD.
% - Add surface drift to ssd. We made some progress in
% Writing SSD drift and diffusion code on GPU using the techniques we developed here. This will allow running a large charge cloud must faster in SSD
% \subsection{Improvements to Neural Network based Electronics Modeling}
%- tail slope does not match for ATN output with data
%- we also tried adding electronics effect to simulations and training with 1-1 waveforms and the results did not match
% - use distribution loss such a Wasserstein metric for known parameters such astail slope, current amplitude, drift time, etc.

\subsection{Improving interpolation for {\ehd} maps}
Until now, we have been using cubic interpolation to build activeness maps in {\ehd}. Although this method is a good approximation, it does not always capture the sudden changes in activeness near the passivated surface. This is because the charge-collecting efficiency varies much more rapidly in the near-surface areas where surface effects dominate than in the bulk of the detector. Thus, the cubic interpolated tend to have sharp edges close to the surface.

One way to address this is to adopt interpolation that accounts for local gradients or discontinuities in the activeness function. Techniques like radial basis function (RBF) interpolation could be used to refine the interpolation in high-variation regions while maintaining a coarser interpolation in the bulk. Another possibility is to perform a study on the activeness as a function of detector type, energy, and surface charge to determine the optimal sampling of the detector.

\subsection{Future for Pulse Shape Simulations}

The future development in pulse shape simulations for Germanium detectors will be centered around {\ssd} . Although {\ehd} successfully captures diffusion, self-repulsion, and surface charge effects, it also makes several assumptions in 2-D simulations of a 3-D problem that {\ssd} can, in principle, avoid. The fully three-dimensional approach of {\ssd} provides a more complete picture of electric fields and charge drift, and it also provides the ability to model the effects of the surroundings of a detector. The programmer is written in \texttt{Julia} which provides a natural way to parallelize the calculations.

The techniques developed in {\ehd} such as GPU-based drift and diffusion and surface drift can be used to improve {\ssd}. Although {\ssd} already uses GPU for field calculations, performing drift and diffusion on GPUs will enable simulating large charge cloud surface events much faster in {\ssd}. We performed some initial {\ssd} simulations with surface drift and the results looked promising.

 % The goal is to merge {\ehd}’s detailed physics features—like non-spherical charge clouds near the passivated surface - with the robust 3D framework of {\ssd}, allowing faster and more comprehensive pulse shape simulations without the simplifying assumptions that limit {\ehd}.

\subsection{Improvements to Neural Network based Electronics Modeling}
The CPU-Net was quite effective at translating simulated waveforms to data-like pulses, while matching the current amplitude and drift time distribution. However, the CycleGAN training can be quite unstable and requires fine-tuning of the several hyperparameters to get the right balance. This can be seen in figure \ref{ch8_fig_tail_slope_sim}, where the distribution of data and ANT output waveform do not match. We also noticed that the noise added by CPU-Net is not always consistent with what we see in the data. To test both effects, we added pole zero decay and pink noise to the simulated waveform. This gave us a direct one-to-one data set for validation. We trained CPU-Net with simulations plus electronics effects as data. After training we observed that while the ATN output waveform is active aligned with the data waveform better, there were still a significant difference between them. This shows that the model learns the electronics response to an extent, just enough to fool the discriminator.

One way to improve the model is to incorporate distribution-based loss metrics, such as the Wasserstein distance, into loss optimization. This can be done for known parameters such as the tail slope, current amplitude, and drift time. This forces the model to learn the correct distribution during training. We would not know how the model works for unknown electronics parameters; however, proving the known parameters can give the model a first guess to improve upon. Combining the Wasserstein metrics with cycle consistency and identity consistency could help the generator learn the feature better.

Another approach would be to replace CycleGAN with an advanced generative model such as the diffusion model \cite{2020arXiv200611239H} or the Transformer model \cite{Transformer}. The diffusion model gradually adds Gaussian noise to the data pulse in a Markov chain. Then it trains a network to reverse the chain. This allows for finer control over the learned distribution. The advantage of the method is that we can control the noise being encoded. Thus, using known waveform shape parameter priors, such as tail slop, current amplitude, and drift times, we can guide the model to learn the correct translations of simulated pulses.

\label{chap:conclusion}